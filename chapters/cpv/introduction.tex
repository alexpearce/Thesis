\chapter{Introduction}
\label{chap:cpv:introduction}

Direct \CP\ violation has only been observed in the
kaon~\cite{Christenson:1964fg,Batley:2002gn} and \PB meson
systems~\cite{Aubert:2001nu,Abe:2001xe,Aaij:2012kz,Aaij:2013iua}.
The charm system is the only one in which the \ac{SM} predicts \CP\ violation
but where no \CP\ symmetry breaking has been observed.
Particular phenomenological focus has rested on the \PDzero\
meson~\cite{Grossman:2006jg}, the lightest charm hadron.
The most sensitive experimental probes of direct \CP\ violation in \PDzero
decays have been measurements of the difference between the rates of direct
\CP\ violation in two \ac{SCS} decays of the \PDzero meson,
$\decay{\PDzero}{\KmKp}$ and \pimpip.

The rate of direct \CP\ violation in the decay of any particle $P$ to a final
state $f$ is parameterised as
\begin{equation}
  \ACP(\decay{P}{f}) = \frac{%
    \Gamma(\decay{P}{f}) - \Gamma(\decay{\bar{P}}{\bar{f}})
  }{%
    \Gamma(\decay{P}{f}) + \Gamma(\decay{\bar{P}}{\bar{f}})
  }.
  \label{eqn:cpv:introduction:acp}
\end{equation}
For \decay{\PDzero}{\hmhp} decays, this is expected to be of the order of
\num{e-3} in the \ac{SM}~\cite{Grossman:2006jg}, and as such is sensitive to
contributions from physics \acl{BSM}.

The \KmKp\ and \pimpip\ final states transform into one another under the 
$U$-spin transformation, exchanging $\Pdown \rightleftharpoons \Pstrange$ 
quarks. In some models the sign of $\ACP(\decay{\PDzero}{\KmKp})$ is predicted 
to be opposite to that of $\ACP(\decay{\PDzero}{\pimpip})$ through $U$-spin 
symmetry arguments~\cite{Grossman:2006jg}, and so the difference between the 
two is particularly sensitive to direct \CP\ violation
\begin{equation}
  \dACP = \ACP(\decay{\PDzero}{\KmKp}) - \ACP(\decay{\PDzero}{\pimpip}).
  \label{eqn:cpv:introduction:dacp_pure}
\end{equation}

After an initial measurement of \dACP\ providing evidence of \CP\ violation in
charm decays~\cite{Aaij:2011in,Lenz:2013pwa}, generating much interest in the
theory community~\cite{Lenz:2013pwa}, more precise measurements of \dACP\ did
not confirm the initial evidence~\cite{Aaij:2014gsa,Aaij:2016cfh}.
The current world average is consistent with the hypothesis of no direct \CP\
violation in \PDzero decays~\cite{Amhis:2014hma}, being\footnotemark
\begin{equation}
  \dACP = \SI{-0.252 \pm 0.104}{\percent}.
  \label{eqn:cpv:introduction:dacp_world_average}
\end{equation}
This is to be compared to theoretical predictions which are uniformly below the 
percent level, but in which large theoretical uncertainties prevent a 
consistent number.

\footnotetext{%
  The world average is not strictly of \dACP, but of
  $\Delta{a}_{\CP}^{\text{dir}}$, which includes effects in \emph{indirect}
  \CP\ violation, which arise to the possibility of \CP\ violation in
  \PDzero-\APDzero mixing~\cite{Gersabeck:2011xj}.
}

Charm baryons, the lightest of which is the \PLambdac~($\Pup\Pdown\Pcharm)$,
are an additional system in which to search for \CP\ violation.
The first evidence for \CP\ violation in baryon decays was found only
recently~\cite{Aaij:2016cla}, using the decays of the lightest beauty baryon,
the \PLambdab~($\Pup\Pdown\Pbottom$).
An analogous measurement of \dACP\ using \PLambdac\ decays, to that made with
\PDzero decays, is searching for direct \CP\ violation in the decays \LcTopKK\
and \LcToppipi.
The dynamics of these \ac{SCS} decays, which can only be fully parameterised by
a five-dimensional phase space~\cite{Aitala:1999uq}, are currently poorly
understood~\cite{Bigi:2012ev,PDG2014}, and so any experimental input can be
useful to guide and constrain the theory.
The previous generation of heavy flavour collider experiments, such as \belle\
and \babar, collected small samples of \PLambdac~\cite{Seuster:2005tr}, whereas
the large charm cross-section at \lhcb\ enables \CP\ violations measurements at
the percent level or lower.

The large sample of \Pbottom hadron decays collected by \lhcb\ can also be used
to study \CP\ violation in \PLambdac\ decays.
The long \PLambdab\ lifetime, around seven times greater than that of the
\PLambdac, can be exploited to provide efficient background rejection.
The semileptonic decay chain \LbToLcmuX\ enables the usage of the \lhcb\ muon
triggers, described in \cref{chap:intro:lhcb:trigger}, requiring a single, high
\pT\ muon whose trajectory is significantly displaced from the \ac{PV}.
Using measurements of the \LbToLcmuX\ and \LcTophh\ branching
fractions~\cite{PDG2014,Ablikim:2016tze}, and measurements of the \PLambdab\
cross-sections in \pp\ collisions~\cite{Aaij:2015fea}, and assuming a total
reconstruction and selection efficiency of around \SI{1}{\percent}, \lhcb\
should have collected of the order of \num{e4} \ppipi\ decays and \num{e3} \pKK
decays during \runone.
The statistical uncertainty on a measurement of \dACP\ will then be dominated
by the \pKK\ sample size, having a precision on the order of
\SI{1}{\percent}.

The analysis described in this \lcnamecref{chap:cpv} measures \dACP\ using
\LcTopKK\ and \LcToppipi decays, reconstructed in association with a high-\pT\
muon assumed to originate from the decay of a \PLambdab.
The data were collected with the \lhcb\ detector in 2011 and 2012,
corresponding to an integrated luminosity of \SI{3}{\per\femto\barn}.
An overview of the analysis will be given in the remainder of this
\lcnamecref{chap:cpv:introduction}.

\section{Analysis overview}
\label{chap:cpv:introduction:overview}

Rather than measuring the relative difference of decay rates in
\cref{eqn:cpv:introduction:acp}, it is simpler to measure the asymmetry in the
observed yields $N$
\begin{equation}
  \ARaw(f) = \frac{N(f) - N(\bar{f})}{N(f) + N(\bar{f})},
  \label{eqn:cpv:introduction:araw}
\end{equation}
where here, and throughout this \lcnamecref{chap:cpv}, the origin of the final
state $f$ from a \PLambdac\ decay is implicit.\footnotemark\
This is not necessarily equal to the decay asymmetry in
\cref{eqn:cpv:introduction:acp} as it may contain contributions from associated
production and detection asymmetries.
The complete formalism and experimental treatment of this will be discussed in
\cref{chap:cpv:theory}.

\footnotetext{%
  It is assumed that charge is conserved, so a final state $\bar{f}$ implies an
  initial state \APLambdac.
}

Any theoretical computation of \ACP\ or \dACP\ for \LcTophh\ decays requires
some model for the five-dimensional \phh\ phase space, as it may be that the
strength of \CP\ violation varies across it.
The selection and reconstruction of the \PLambdac\ candidates may sculpt this
distribution, and so the experimental efficiency as a function of the phase
space must be computed.
The phase space definition is given \cref{chap:cpv:theory:phsp}.

After correcting for any possible background asymmetries with a weighting
procedure, and for the phase space efficiencies computed using simulation, the
\PLambdac\ and \APLambdac\ yields are measured simultaneously for each mode
with \chisq\ fits.
The fits are performed in such a way that the true value of \ARaw\ is hidden to
avoid experimental bias, and as such the fits are said to be `blinded'.

The description of the data by the fits, any remaining background asymmetries,
and \PLambdac\ decays originating from sources other than \PLambdab\ decays are
considered in the context of systematic uncertainties.
Studies are performed, and deviations from the nominal, measured value of
\dACP\ that cannot be corrected for will be assigned as systematic
uncertainties on the measurement.

\subsection{Part structure}
\label{chap:cpv:introduction:overview:structure}

The contributions of the various production and detection asymmetries to \ARaw
will be derived and enumerated in \cref{chap:cpv:theory}.
The parameterisation of the five-dimensional phase space will also be given,
which is used when constructing the phase space efficiency model.
The collision data and the simulation used in the analysis are described in
\cref{chap:cpv:data}, and the reconstruction and selection of the real data
will be presented in \cref{chap:cpv:selection}.
The extraction of the signal yields and of the kinematic distributions of the
signal from the fully selected data is presented in
\cref{chap:cpv:prelim_fits}, and then those signal distributions are equalised
between the two \PLambdac\ decay modes, discussed in
\cref{chap:cpv:kinematic_weighting}, to remove background asymmetries such as 
the \PLambdab\ production asymmetry.
The resulting kinematic weights are combined with reconstruction and selection
efficiencies computed as a function of \phh\ phase space, whose method of
computation is shown is \cref{chap:cpv:phsp}, after which the combined weights
enter the blinded \chisq\ fits that are used to measure \ARaw, described in
\cref{chap:cpv:araw}.
The combination of $\ARaw(\pKK)$ and $\ARaw(\ppipi)$ to measure \dACP\ is given
in \cref{chap:cpv:results}.
Methods for, and the results of, various systematic studies are given in
\cref{chap:cpv:syst}.
Finally, a summary is presented in \cref{chap:cpv:summary}.
