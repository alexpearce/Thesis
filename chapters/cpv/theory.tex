\chapter{Formalism}
\label{chap:cpv:theory}

\subsection{\texorpdfstring{\ACP\ and \dACP}{ACP and delta ACP}}
\label{chap:cpv:theory:acp}

The size of the asymmetry in the decay of a \PLambdac\ baryon 
($\Pup\Pdown\Pcharm$) to some final state $f$ is given by the parameter \ACP\
\begin{equation}
  \ACP(f) = \frac{\Gamma(f) - \Gamma(\bar{f})}{\Gamma(f) + \Gamma(\bar{f})},
  \label{eqn:cpv:theory:acp}
\end{equation}
where $\Gamma(f)$ is the decay rate of \decay{\PLambdac}{f}, and $\bar{f}$ is 
the charge conjugate of $f$.\footnotemark
If the process \decay{\PLambdac}{f} occurs at a higher rate than 
\decay{\APLambdac}{\bar{f}}, for example, then $\ACP > 0$.
Measuring the decay rate $\Gamma$ experimentally requires counting the number 
of \PLambdac\ produced, which is hard to measure precisely, as shown in 
\cref{chap:prod}.
It is simpler to count only the number $N$ of \emph{reconstructed} \PLambdac\ 
decays, and to form an asymmetry parameter as
\begin{equation}
  \ARaw(f) = \frac{N(f) - N(\bar{f})}{N(f) + N(\bar{f})}.
  \label{eqn:cpv:theory:araw}
\end{equation}
This is the \emph{raw} asymmetry as is it not necessarily equal to \ACP, being 
contaminated by possible detector and production effects.

As discussed in \cref{chap:cpv:introduction}, this analysis reconstructs 
\PLambdac\ decays originating from \decay{\PLambdab}{\PLambdac\Pmuon X}\ 
decays.
The value of \ARaw\ can then include the effects of the \PLambdab/\APLambdab\ 
production asymmetry, the \Pmuon/\APmuon detection asymmetry, and the detection 
asymmetry of the \PLambdac\ final state $f$/$\bar{f}$, if they are non-zero.

% citation for non-zero Lb prod asym?
The \PLambdab\ production asymmetry is predicted to be non-zero due to the \pp\ 
collisions provided by the \ac{LHC}.
The higher `availability' of matter quarks compared to antimatter quarks 
results in higher production cross sections of matter hadrons compared to that 
of antimatter.
% TODO should make a statement like the one for the det. asyms., that more Lb 
% means more Lc means more f, looking like a non-zero ACP
The two detection asymmetries are also expected to be non-zero, due to the 
known differences between the \PKminus/\PKplus, \Ppiminus/\Ppiplus, and 
\Pproton/\APproton cross-sections with matter (the material which makes up the 
\lhcb\ detector).
A higher \Pmuon reconstruction efficiency with respect that for \APmuon would 
result in a higher proportion of \PLambdac\ decays being reconstructed, and 
similarly so for the reconstruction efficiency of the final states $f$ and 
$\bar{f}$.
If these effects were not taken into account, one might incorrectly conclude 
that a measured difference between the number of reconstructed final states $f$ 
and $\bar{f}$, that is \ARaw, means a non-zero value of \ACP\@.
% As the $\Php\Phm$ meson pair is charge-symmetric, the detection asymmetry of 
% the final state is effectively the proton detection asymmetry.

The number of reconstructed \LcTof\ decays can be expressed as the product of 
several effective probabilities
\begin{equation}
  N(f) = \prob(\PLambdab)\cdot
         \Gamma(\LbToLcmuX)\cdot
         \Gamma(\LcTof)\cdot
         \eff(\Pmuon)\cdot
         \eff(f),
  \label{eqn:cpv:theory:yield}
\end{equation}
and similarly for $N(\bar{f})$, where $\prob(\PLambdab)$ is the probability of 
producing a $\PLambdab$ baryon given a \pp\ collision, $\Gamma(\decay{A}{B})$ 
is the decay rate of the particle $A$ to final state $B$, and $\eff(\Pmuon)$ 
and $\eff(f)$ are the muon and $\PLambdac$ final-state detection efficiencies.
% TODO can we back up this assumption?
It is assumed that the \LbToLcmuX decay is \CP-symmetric, but that all other 
factors may not be, for example $\eff(\Pmuon) \neq \eff(\APmuon)$.
In the following, it will be shown how a variable can be constructed to 
eliminate the various background asymmetries, leaving only the asymmetry of 
interest, \ACP\@.

To express \ARaw\ in terms of all of its component asymmetries, it is useful to 
first define the notation of a general asymmetry parameter $X$, which describes 
the asymmetry between two quantities $x$ and $\bar{x}$
\begin{equation}
  X = \frac{x - \bar{x}}{x + \bar{x}}.
  \label{eqn:cpv:theory:generic_asym}
\end{equation}
This general form can be rearranged as
\begin{align}
  x &= \frac{1}{2}(x + \bar{x})(1 + X),\label{eqn:cpv:theory:asym_form_one}\\
  \bar{x} &= \frac{1}{2}(x + \bar{x})(1 - X).\label{eqn:cpv:theory:asym_form_two}
\end{align}
An asymmetry parameter like that in \cref{eqn:cpv:theory:generic_asym} can be defined for each of the relevant terms in 
\cref{eqn:cpv:theory:yield}
\begin{align*}
  \APLb(f) &= \frac{%
    \prob(\PLambdab) - \prob(\APLambdab)
  }{%
    \prob(\PLambdab) + \prob(\APLambdab)
  },\\
  \ADmu(f) &= \frac{%
    \eff(\Pmuon) - \eff(\APmuon)
  }{%
    \eff(\Pmuon) + \eff(\APmuon)
  },\\
  \ADf(f)  &= \frac{%
    \eff(f) - \eff(\bar{f})
  }{%
    \eff(f) + \eff(\bar{f})
  }.
\end{align*}
% TODO should say why
Each parameter is, at least implicitly, dependent on the detected final state 
$f$.

Substituting in Equation~\ref{eqn:cpv:theory:yield} to 
Equation~\ref{eqn:cpv:theory:araw}
\begin{equation*}
  \ARaw(f) = \frac{%
    \prob(\PLambdab)\Gamma(f)\eff(\Pmuon)\eff(f) - 
    \prob(\APLambdab)\Gamma(\bar{f})\eff(\APmuon)\eff(\bar{f})
  }{%
    \prob(\PLambdab)\Gamma(f)\eff(\Pmuon)\eff(f) + 
    \prob(\APLambdab)\Gamma(\bar{f})\eff(\APmuon)\eff(\bar{f})
  },
\end{equation*}
and then substituting each quantity for its equivalent form as in 
\cref{eqn:cpv:theory:asym_form_one,eqn:cpv:theory:asym_form_two}, all factors 
of $\sfrac{1}{2}$ and all factors of the form $(x - \bar{x})$ cancel, leaving
% TODO format this disgustingly long equation
\begin{equation*}
  \ARaw(f) = \frac{%
    \APLb\ADmu\ADf + \APLb\ADmu\ACP + \APLb\ADf\ACP + \ADmu\ADf\ACP + \APLb + 
    \ADmu + \ADf + \ACP
  }{%
    1 + \APLb\ADmu + \APLb\ADf + \APLb\ACP + \ADmu\ADf + \ADmu\ACP + \ADf\ACP + 
    \APLb\ADmu\ADf\ACP
  }.
\end{equation*}
Assuming that the individual asymmetries are small, of the order of 
\SI{1}{\percent}, the product of two or more asymmetries is negligible with 
respect to the leading order, and so
\begin{equation}
  \ARaw(f) \approx \ACP(f) + \APLb + \ADmu + \ADf.
\end{equation}

\footnotetext{%
  It is assumed that charge is conserved, so a final state $\bar{f}$ implies an 
  initial state \APLambdac.
}

% TODO haven't said that ADf is mostly proton det. asym. yet, so it seems 
% bizarre that ADf would cancel in the different between pKK and ppipi
By assuming that these background asymmetries are mode-independent, that is to 
say $\AD(f) = \AD(g)$ and $\AP(f) = \AP(g)$, we can take two \PLambdac\ decay 
modes and measure the \emph{difference} in their raw asymmetries, eliminating 
the background asymmetries by construction.
This analysis reconstructs the \ac{SCS} decays \LcTopKK\ and \LcToppipi, giving
\begin{align}
  \dACP &= \ARaw(\pKK) - \ARaw(\ppipi),\label{eqn:cpv:theory:dacp}\\
        &\approx \ACP(\pKK) - \ACP(\ppipi)\nonumber.
\end{align}

The assumption that the production and detection asymmetries in \cref{} are 
mode independent is not true in general.
By making a more reasonable assumption that these background asymmetries are 
dependent only on the kinematics of the representative particles, this is made 
clear by considering that different final states will in general have different 
reconstruction and selection efficiencies as a function of \PLambdac\ 
kinematics.
Two samples with different \PLambdac\ kinematics will also have different 
\PLambdab\ and muon kinematics, and hence there will be net production and 
detection asymmetries between them.

The assumption that the background asymmetries depend only on particle 
kinematics is not unreasonable; the production asymmetry, for example, is a 
difference in cross sections, which are generally parameterised by the 
kinematics of the produced particle (some combination of momentum \ptot, 
transverse momentum \pT, and pseudorapidity \Eta), and of the properties of the 
colliding beams (which are independent of the produced particle).
Similarly, a detection asymmetry describes the differences of material 
interactions between matter and antimatter, and these are dependent on the 
momentum of the particle in question and, assuming a non-uniform material 
distribution, its flight path.
These can be described by some combination of the kinematic variables 
previously given.

Given a sample of \PLambdac\ decays with equal kinematics, it is reasonable to 
assume that the kinematics of the muon and of the \PLambdab\ are independent of 
the decay modes present in that sample.
Hence, for a sample of \LcTopKK and \LcToppipi decays, after calibrating the 
\PLambdac\ kinematics to match one another, we posit that the \PLambdab\ and 
\Pmuon kinematics will match between the two modes, and hence the background 
asymmetries \AP\ and \ADmu\ will cancel in the difference \dACP\@.
The assumption of matching \PLambdab\ and \Pmuon kinematics will have to be 
proven explicitly.

Finally, it is required to show that the kinematics of the protons between the 
\pKK\ and \ppipi\ also agree.
It is not so obvious, as before, that these variables would a priori match, and 
indeed it seems sensible to assume that they would not, and so we must 
demonstrate explicitly whether it is the case.
If it is not, we cannot cannot imagine any clever calibration which would make 
it so, and consequently a proton detection asymmetry would be required which, 
as of this writing, does not exist.\footnotemark

\footnotetext{%
  There have been preliminary studies performed in to the feasibility of a 
  proton detection asymmetry measurement~\cite{Pearce:2014bb}.
}

\section{Decay phase space}
\label{chap:cpv:theory:phsp}

% TODO
