% Manchester guidelines say the information in the description environment 
% should be provided on the abstract page, which should not be more than a 
% single side of A4 paper, hence the short body
\chapter{Abstract}

\begin{SingleSpace}

\begin{description}
  \item[University] The University of Manchester
  \item[Candidate] Alex Pearce
  \item[Degree Title] Doctor of Philosophy
  \item[Thesis Title] \thetitle
  \item[Date] October 2016
\end{description}

This thesis presents two measurements made using data collected by the \lhcb\ 
detector, operating at the \acl{LHC} accelerator at the CERN particle physics 
laboratory.
The first is a measurement of the production rates of promptly produced 
\PDzero, \PDplus, \PDsplus, and \PDstarp\ open charm mesons, using data 
collected in 2015 at a proton-proton centre-of-mass energy of \sqrtseq{13}.
The second is a search for direct \CP\ violation in two three-body decays of 
the \PLambdac\ charm baryon, \pKK\ and \ppipi, using data collected in 2011 at 
\sqrtseq{7} and in 2012 at \sqrtseq{8}.
For each measurement, motivation and context are given from the standpoint of 
improving the theoretical understanding of the \acl{SM} and searching for signs 
of physics that cannot be explained by it, and then the various statistical 
analysis techniques used to extract physical quantities from the data are 
explained.
The systematic limitations of the method are explored and quantified, and then 
the results are presented.

\end{SingleSpace}
