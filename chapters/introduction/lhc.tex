\chapter{The \acl{LHC}}
\label{chap:intro:lhc}

As high-energy particle physics has progressed over the last century, the 
meaning of `high energy' has continuously evolved.
Today, high energy refers to the \acl{LHC}~\cite{Bruning:2004ej}, a circular 
collider which began operating at a proton-proton centre of mass energy of 
\sqrtseq{7} in 2010, and in 2015 reached \sqrtseq{13}.
This section will describe the machine, the experiments which it services, and 
the changes in operation throughout its lifetime so far.

In order to produce collisions, bunches of \num{1.2e11} protons are obtained 
through the ionisation of hydrogen gas, and are then accelerated through a 
chain of progressively larger, more powerful accelerators before being injected 
into the \ac{LHC}.
During nominal operation, a single \ac{LHC} `fill' consists of injecting 2808 
bunches into the machine, each of which is about \SI{30}{\centi\metre} long, 
separated from the surrounding bunches by about \SI{7}{\metre}, completing a 
full revolution of the ring at a rate of \SI{11.246}{\mega\hertz}.

The beams are steered around the ring by superconducting dipole magnets, cooled 
to \SI{1.9}{\kelvin}, and are focused by quadrupole and other, higher-order, 
magnets.
The beams are made to collide at four \acp{LHCIP}, before which they are 
focused further, from the usual circulating bunch diameter of 
\SI{0.2}{\milli\metre} to a width on the order of microns, such that they can 
collide.

For a collision experiment, the rate $N$ at which a particular process occurs 
per second is given by
\begin{equation}
  N = \lumi\xsec,
  \label{eqn:intro:lhc:lumi_xsec}
\end{equation}
where \lumi\ is the instantaneous luminosity of the accelerator and \xsec\ is 
the cross-section of the process.
At $\sqrts = \SI{13}{\TeV}$, the visible proton-proton interaction 
cross-section is around $\xsecppvis = 
\SI{70}{\milli\barn}$~\cite{Aaboud:2016mmw,CMS:2016ael}.
Given that this number is fixed by the beam energy and colliding particle type, 
the \ac{LHC} is designed to maximise the luminosity \lumi, which can be 
expressed as
\begin{equation}
  \lumi = \frac{N_{\text{p}}^{2}N_{\text{b}}\revfreq\gamma}{4\pi\epsilon\beta^{*}}F,
  \label{eqn:intro:lhc:inst_lumi}
\end{equation}
where $N_{\text{p}}$ is the number of particles per bunch; $N_{\text{b}}$ is 
the number of bunches per beam; \revfreq\ is the bunch revolution frequency; 
$\gamma$ is the Lorentz factor; $\epsilon$ is the beam emittance during 
circulation, a measure of the beam size transverse to the beam direction; 
$\beta^{*}$ is a measure of the transverse beam size at a particular 
\ac{LHCIP}; and $F$ accounts for non-zero crossing angles between the two beams 
at an \ac{LHCIP}.
Depending on the experiment, the \betastar\ parameter can be tuned to provide a 
specific average number $\nu$ of collisions per bunch crossing.
The design luminosity at the \atlas\ and \cms\ \acp{LHCIP} is
\SI{1e34}{\per\square\centi\metre\per\second}, which results in a rate of 
visible proton-proton interactions of
\begin{equation}
  N_{\text{vis.}} = \lumi\xsecppvis
                  = \SI{700}{\mega\hertz}.
  \label{eqn:intro:lhc:nppvis}
\end{equation}

\subsection{Running periods}

The \ac{LHC} first began running in 2010 at a proton-proton centre-of-mass 
energy of $\sqrts = \SI{7}{\TeV}$, and this continued until the end of 2011.
In 2012, the centre-of-mass energy was increased to $\sqrts = \SI{8}{\TeV}$, 
before the machine was shut down at the end of 2012.
This marked the end of \runone\ and the beginning of \acf{LS1}, during which 
the accelerator was upgraded in order to prepare the machine for running at 
$\sqrts = \SI{13}{\TeV}$.
This new phase of operation, \runtwo, began in mid-2015, and is expected to 
continue until the end of 2018.
