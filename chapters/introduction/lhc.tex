\chapter{The \acl{LHC}}
\label{chap:intro:lhc}

As high-energy particle physics has progressed over the last century, the 
meaning of `high energy' has continuously evolved.
Today, high energy refers to the \acl{LHC}~\cite{Bruning:2004ej}, a circular 
collider which begin operating at a proton-proton centre of mass energy of 
\sqrtseq{7} in 2010, and in 2015 reached \sqrtseq{13}.

In order to produce collisions, bunches of 120 billion protons are obtained 
through the ionisation of hydrogen gas, and are then accelerated through a 
chain of progressively larger, more powerful accelerators before being injected 
into the \ac{LHC}.
During nominal operation, a single \ac{LHC} `fill' consists of injecting 2808 
bunches into the machine, each of which is about \SI{30}{\centi\metre} long, 
separated from the surrounding bunches by about \SI{7}{\metre}, completing a 
full revolution of the ring at a rate of \SI{11.2}{\mega\hertz}.

The beams are steered around the \ac{LHC} by superconducting dipole magnets, 
cooled to a temperature of \SI{1.9}{\kelvin}, and are focused by quadrupole and 
other, higher-order, magnets.
The beams are made to collide at four \acp{LHCIP}, before which they are 
focused further, from the usual bunch diameter of \SI{0.2}{\milli\metre} while 
circulating, to a width on the order of microns, such that they can interact 
and collide.

For a collision experiment, the number of times $N$ a particular process occurs 
per second is the product of two quantities
\begin{equation}
  N = \lumi\xsec,
  \label{eqn:intro:lhc:lumi_xsec}
\end{equation}
where \lumi\ is the instantaneous luminosity of the accelerator and \xsec\ is 
the cross-section of the process in question.
At $\sqrts = \SI{13}{\TeV}$, the largest cross-section seen at the \ac{LHC} for 
a visible proton-proton interaction is around $\xsecppvis = 
\SI{70}{\milli\barn}$~\cite{Aaboud:2016mmw,CMS:2016ael}.
Given that this number is fixed by the beam energy and colliding particle type, 
the \ac{LHC} is designed to maximise the luminosity \lumi, which can be 
expressed as
\begin{equation}
  \lumi = \frac{N_{\text{p}}^{2}N_{\text{b}}\revfreq\gamma}{4\pi\epsilon\beta^{*}}F,
  \label{eqn:intro:lhc:inst_lumi}
\end{equation}
where $N_{\text{p}}$ is the number of particles per bunch; $N_{\text{b}}$ is 
the number of bunches per beam; \revfreq\ is the bunch revolution frequency; $\gamma$ 
is the Lorentz term as in \cref{eqn:intro:sm:lorentz_factor}, where $\beta 
\approx 1$ at the \ac{LHC}; $\epsilon$ is the beam emittance during 
circulation, a measure of the beam size transverse to the beam direction; 
$\beta^{*}$ is a measure of the transverse beam size at a particular 
\ac{LHCIP}; and $F$ accounts for non-zero crossing angles between the two beams 
at an \ac{LHCIP}.
Depending on the experiment, the \betastar\ parameter can be tuned to provide a 
specific average number $\nu$ of collisions per bunch crossing.
Maximising the luminosity is limited by the technological and monetary 
constraints present at the time of construction: increasing the number of 
bunches or particles per bunch increases the beam current, requiring stronger 
magnetic fields to steer and focus the beams, and likewise for decreasing the 
transverse bunch profile.
The design luminosity of the \ac{LHC} at the \atlas\ and \cms\ \acp{LHCIP} is
\SI{1e34}{\per\square\centi\metre\per\second}, which results in a rate of 
visible proton-proton interactions of
\begin{equation}
  N_{\text{vis.}} = \lumi\xsecppvis
                  = \SI{700}{\mega\hertz}.
  \label{eqn:intro:lhc:nppvis}
\end{equation}

As the most powerful particle accelerator in the world, the \ac{LHC} is itself 
an experiment, pushing the boundaries of what is capable with a particle 
collider.
The real purpose of the \ac{LHC}, however, is to provide the experiments with 
collisions, so that they can find the next big discovery.

\section{Interaction points}
\label{chap:intro:lhc:ips}

A set of eight access points provide entry to the underground tunnel which 
contains the \ac{LHC}.
Four of these points lead to large underground caverns that house particle 
detectors, whilst the others are used for accelerator maintenance.

The \atlas\ detector~\cite{Aad:2008zzm} is at point 1, and point 5 houses the 
\cms\ detector~\cite{Chatrchyan:2008aa}.
Each of these are cylindrical detectors centred around the beam-beam 
interaction region, with \atlas\ being \SI{44}{\metre} long and \SI{25}{\metre} 
in diameter, and \cms\ being \SI{22}{\metre} long and \SI{15}{\metre} in 
diameter.
% TODO: check this statement is true, does the Higgs really do that
They are designed to discover and study the properties of the Higgs boson 
\PHiggs, the particle the generates the mass of the fundamental particles, and 
to study the physics of \ac{QCD} and \ac{EW} interactions.
The signals they search for are generally high-energy and high-momentum 
electrons, muons, photons, and jets, collimated clusters of charged and neutral 
hadrons.
The high momentum of these signals is due to the large masses of the particles 
\atlas\ and \cms\ search for, which include \PW, \PZ, and \PHiggs bosons, with 
masses between 80 and \SI{125}{\GeVcc}, and particles predicted by \ac{BSM} 
theories, which can be in the order of \SI{1}{\TeV}.

Given the design luminosity at points 1 and 5 of 
\SI{1e34}{\per\square\centi\metre\per\second}, and that the rate of \PHiggs 
production in proton-proton collisions is of the order of tens of 
picobarn~\cite{Khachatryan:2016vau}, $\xsec_{\PHiggs} \sim \SI{10}{\pico\barn} 
= \SI{1e-35}{\per\square\centi\metre}$, translating to around 0.1 Higgs bosons 
being produced every second using \cref{eqn:intro:lhc:lumi_xsec}.
The cross-section for new particles predicted by \ac{BSM} theories are even 
lower than that for the Higgs, and so \atlas\ and \cms\ must collect large 
samples of data to make statistically significant discoveries, requiring high 
beam luminosities to achieve it.
As a consequence of the high luminosity, the detectors observe a large number 
of proton-proton interactions per bunch-bunch crossing, around 20 by 
\cref{eqn:intro:lhc:nppvis}, and this leads to a large amount of data being 
produced per crossing, approximately \SI{1}{\mega\byte}.
This would amount to \SI{40}{\tera\byte} of data per second of \ac{LHC} 
running, but most of this cannot be saved due to monetary and technological 
constraints.
Instead, on order of 100 crossings per second are saved for subsequent 
analysis.

In addition to the nominal proton running, the \ac{LHC} also runs with 
configurations of ion beams for about one-eighth of its operating time, 
specifically with two beams of lead~(Pb) ions in the lead-lead configuration, 
and with one beam of protons and another beam of lead ions in the proton-lead 
configuration.
The \alice\ detector~\cite{Aamodt:2008zz}, located at point 2, focuses on 
studying these lead-lead and proton-lead collisions.
Specifically, \alice\ measures the properties of the quark-gluon plasma that 
forms in ion collisions, which is a hot, dense state of matter that was 
dominant in the early stages of the universe.
Such measurements provide important input to the theoretical understanding of 
macroscopic behaviour driven by the strong force, such as in the quark-gluon 
plasma, where predictions are complex to compute.

Finally, the \lhcb\ detector~\cite{Alves:2008zz} is located at point 8.
It will be described in detail in \cref{chap:intro:lhcb}.

\section{Running periods}

The \ac{LHC} first began running in 2010 at a proton-proton centre-of-mass 
energy of $\sqrts = \SI{7}{\TeV}$, and this continued until the end of 2011.
In 2012, the centre-of-mass energy was increased to $\sqrts = \SI{8}{\TeV}$, 
before the machine was shut down at the end of 2012.
This marked the end of \runone\ and the beginning of \ac{LS1}, during which the 
accelerator was upgraded in order to prepare the machine for running at $\sqrts 
= \SI{13}{\TeV}$.
This new phase of operation, \runtwo, began in mid-2015, and is expected to 
continue until the end of 2018.

The trend of increasing beam energies is driving towards reaching the design 
energy of \SI{7}{\TeV} per beam, giving $\sqrts = \SI{14}{\TeV}$.
Higher beam energies are generally good for the experiments, as the 
cross-sections of the physical processes of interest increases with \sqrts.
It does come with a price, however, as the rate of other physical processes 
also increases with \sqrts, including those one may not be interested in, which 
are generally much larger than the rare processes of interest.
This means that experiments must continually upgrade their own detectors and 
software in order to suppress to the higher rate of uninteresting events.
The upgrade of the \lhcb\ detector during \ac{LS1} will be described in 
\cref{chap:intro:lhcb:detector:upgrades}.
