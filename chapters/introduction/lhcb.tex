\chapter{The \lhcb\ experiment}
\label{chap:intro:lhcb}

The \lhcb\ experiment is comprised of a collaboration of around one thousand 
scientists and engineers, and a particle detector situated at point 8 of the 
\cern\ \ac{LHC}.
The general goal of the experiment is to explore the area of heavy flavour 
physics, the interactions of charm and beauty quarks and particles that contain 
them.
More specific aims will be given in \cref{chap:intro:lhcb:physics}.
The construction of the detector, which has been optimised to study heavy 
flavour physics produced from proton-proton collisions, will be described in 
\cref{chap:intro:lhcb:detector}.
First, a brief introduction to the mechanisms of a high-energy particle physics 
experiment will be given.

% The main problem in a particle physics experiment is that one does not know 
% every property about what is produced in a collision.
% The design of a particle detector tries to maximise the information one can 
% obtain about the collision by exploiting the interactions of the outgoing 
% particles with the material of the detector.
% Given the information made by the detector, one can then try to infer 
% additional information about the event, including information on particles that 
% did not interact with the detector directly.

% The decay of the \PBs meson to two muons, \decay{\PBs}{\Pmuon\APmuon}, is a 
% good example of the types of processes that \lhcb was designed to detect.
% Due to the nature of the proton-proton interactions, the particles produced 
% within the detector are highly boosted with respect to the laboratory, having 
% velocities close to the speed of light.
% This means the lifetime of the particles in the lab frame are much higher than 
% if they were produced as rest.
% The \ABs meson has a lifetime at rest of \SI{2}{\pico\second}, and so can 
% travel several millimetres in the lab.
% Muons have a lifetime at rest of \SI{2}{\micr\second}, and so can travel much 
% further, tens of metres.
% This is one way to distinguish between particles.
% Another way is through the particle's electric charge.
% The \ABs meson is neutral, whereas the muon is charged, and so will ionise the 
% material of the detector much more readily.
% The ionisation produced in the material can be converted to a electric impulse 
% in a circuit, and several such impulses

% The particles can be categorised in several ways: firstly, the \PBs meson is 
% neutral, whereas the muons are charged; and secondly the \PBs meson is very 
% unstable, with a life time of \SI{2}{\pico\second}, whereas the muons are 
% stable.

\section{Physics goals}
\label{chap:intro:lhcb:physics}

\section{Detector}
\label{chap:intro:lhcb:detector}
