\chapter{The \lhcb\ experiment}
\label{chap:intro:lhcb}

The \lhcb\ experiment comprises a collaboration of around one thousand 
scientists and engineers, and a particle detector situated at point 8 of the 
\cern\ \ac{LHC}.
The general goal of the experiment is to explore the area of \emph{heavy 
  flavour physics}, that is the interactions of charm and beauty quarks and 
particles that contain them.
Specific aims will be given in \cref{chap:intro:lhcb:physics}, and the design 
and construction of the detector will be described in 
\cref{chap:intro:lhcb:detector}.
First, a brief introduction to the mechanisms of a high-energy particle physics 
experiment will be given, such that the particular requirements of a heavy 
flavour experiment can be related to the design of a detector.

Heavy flavour hadrons, such as the \PBs meson, have short lifetimes, and do not 
fly far enough to interact with active elements of a detector before decaying.
By the studying the properties of decay products with longer lifetimes, the 
properties of the unseen heavy flavour hadrons can be inferred, to be compared 
with those predicted by the \ac{SM}.
% Such properties include production and decay rates, lifetimes, and masses.

The momentum four-vector $\fvec{p}$ of a particle is defined by three spatial 
components, equal to components of the momentum three-vector $\vec{p}$, and the 
energy $E$
\begin{equation}
  \fvec{p} = \begin{pmatrix}E\\\vec{p}\end{pmatrix}
           = \begin{pmatrix}E\\p_{x}\\p_{y}\\p_{z}\end{pmatrix}.
  \label{intro:lhcb:four_momentum}
\end{equation}
Both the energy and the momentum can be measured, using techniques which will 
be discussed later.
The mass of every massive particle is unique, and so by knowing the 
four-momentum of a particle, one implicitly knows its identity by knowing the 
mass using the relation
% The energy can either be measured directly, using methods discussed later, or 
% computed using an assumed mass \mass\ and the magnitude of the measured 
% momentum three-vector $\ptot = |\vec{p}|$
\begin{equation}
  E = \sqrt{p^{2} + m^{2}}.
  \label{intro:lhcb:energy}
\end{equation}
To compute the four-momentum of a heavy flavour hadron, in order to know its 
identity, one can sum the four-momenta of the decay products
\begin{equation}
  \fvec{p}_{\text{Parent}} = \sum_{\text{Children}} \fvec{p}_{i}.
  \label{intro:lhcb:four_momentum_sum}
\end{equation}
It is then the task of the experiment to measure the three-momentum and energy 
of all the decay products in order to study heavy flavour.

\subsection{Momentum measurements}

Particles with lifetimes much greater than those of heavy flavour hadrons, such 
as muons, kaons, pions, and protons (denoted \Pmupm, \PKpm, \Ppipm, and 
\Pproton/\APproton), can pass through several layers of active detector 
elements.
The elements can be ionised by the passage of the charged particles, and this 
can be converted into electrical signals.
By studying the geometric pattern of these `hits', one can reconstruct the 
physical path, or `track', the particle made as it travelled through the 
detector.
The set of detector elements used to measure hits is called the tracking 
system.
To measure the three-momentum of the particles, a magnetic field can be applied 
across their path, as shown in \cref{fig:intro:lhcb:tracking}, causing them to 
bend in a circular arc due to the Lorentz force
\begin{equation}
  \vec{F} = q\vec{v}\times\vec{B}
          = \frac{q}{m}\vec{p}\times\vec{B},
  \label{eqn:intro:lhcb:lorentz_force}
\end{equation}
where $\vec{v}$ is the particle velocity three-vector, $q$ is the electric 
charge of the particle, $m$ is its invariant mass, and $\vec{B}$ is the 
magnetic field vector.
As the direction of the particle is already known from the track 
reconstruction, the coordinate system can be aligned such that the magnetic 
field is perpendicular to the momentum direction.
\Cref{eqn:intro:lhcb:lorentz_force} then reduces to a scalar equation for the 
force magnitude $F$, and can be equated to the centripetal force
\begin{equation}
  F = \frac{q}{m}pB = \frac{mv^{2}}{r},
  \label{eqn:intro:lhcb:lorentz_centripetal}
\end{equation}
where $r$ is the radius of the circular path the charged particle will follow 
in the magnetic field.
The momentum can then be computed as
\begin{equation}
  p = qrB.
  \label{eqn:intro:lhcb:momentum_from_magnet}
\end{equation}
In this coordinate system, positively charged particles will bend in a plane in 
the opposite direction to negatively charged particles, and so the `charge' of 
a track can be deduced by observing in which direction it bends in the magnetic 
field.
Neutral particles do not leave hits in the tracking system, and so, as there 
are no known long-lived particles with $q \geq 2$, the magnitude is assumed to 
be 1.

For a given magnetic field strength $B$, higher momentum particles will bend 
less, to the point where the tracking system is not precise enough to be able 
to measure the bending radius, and hence the momentum.
Particles with a low enough momentum will be bent out of the detector entirely.
For a given tracking system, larger bending radii will lead to more precise 
momentum measurements, and so there is a balance between the momentum range a 
detector is able to measure, and the precision to which those measurements can 
be made.

\subsection{Energy measurements}

With the momentum three-vector measured with the tracking system, the energy 
$E$ is left as an unknown.
This can be measured using a calorimeter.
The ionisation of the tracking system discussed previously requires a small 
amount of energy, which is lost from the particle traversing the detector.
A calorimeter takes this process to its limit, being designed to absorb 
particles completely, converting their kinetic energy into a measurable form.
The incident particles lose their energy via particle showers, where 
interactions with the detector material creates additional, lower energy 
particles, and those particles traverse the detector, interact, and create 
further particles with lower energy still.

Calorimeters are placed after the tracking system, as most particles are fully 
absorbed into them, making further measurements of their properties impossible.
Also in contrast to the tracking system, calorimeters can be used to measure 
both neutral and charged particles, and usually a series of individual 
calorimeters is placed one after the other in order to measure different 
particle types in turn, which interact differently with the detector.
Electrons and photons deposit their energy via electromagnetic showers, which 
form due to two processes: high-energy electrons emit photons via 
\emph{bremsstrahlung}; and high-energy photons convert into electron-positron 
pairs.
These two processes alternate back and forth, with $\Pelectron\APelectron$ 
pairs emitting bremsstrahlung photons, which in turn convert into 
$\Pelectron\APelectron$ pairs.
Hadrons, such as protons and charged and neutral pions, produce hadronic 
showers, where the energy loss is due to several processes.
Nuclear interactions, where the hadrons collide with the nuclei in the 
detector, produce additional particles which in turn undergo nuclear 
interactions, creating a cascade.
Hadrons created in this cascade can include \Ppizero and \Peta mesons, which 
decay primarily to final states comprised of electron and photons, and hence 
electromagnetic showers are also created.
The final states of the decays of other particles can include muons and 
neutrinos, which can traverse the calorimeter without interacting further, 
resulting in missing energy.
The length at which electromagnetic and hadronic interactions occur is 
characterised by the interaction lengths \radlen\ and \hadlen\ respectively, 
which depends on the density of the material.
For a given density the hadronic interaction length is much larger than the 
electromagnetic, $\radlen \gg \hadlen$, and so typically an electromagnetic 
calorimeter is placed before a hadronic calorimeter in an experiment, such that 
the electron and photon energies can be measured independently of the hadron 
energies.

The \emph{active} elements of a calorimeter are the materials which converts 
the deposited energy of the particles showers into some measurable quantity, 
either electric charge or light.
The amount of charge or light produced relates to the energy deposited.
The \emph{absorber} elements of a calorimeter are high-density materials that 
reduce the kinematic energy of the incident particles and provide the bulk of 
interaction length.
Homogeneous calorimeters are made solely from active materials, whereas 
sampling calorimeters are constructed from alternating layers of active and 
absorber elements.
Homogeneous calorimeters provide the most accurate measurements, as the entire 
energy deposit can be measured, whereas in a sampling calorimeter some fraction 
is lost to the absorber elements and must be estimated.
The disadvantage of using active elements exclusively is that the suitable 
materials are less dense that the those available solely for absorption, and so 
a larger detector is required for a given energy range that is to be measured.
In practice, hadronic calorimeters are almost exclusively of the sampling type.

\subsection{Particle identification}

If the momentum and energy can be measured precisely, the mass of the particle 
can be deduced with \cref{intro:lhcb:energy}, and hence the particle's identity 
can be known.
As the momentum and energy measurements are subject to experimental resolution, 
there is an uncertainty on the mass measurement.
Alternative \ac{PID} methods are often employed improve this precision.

For charged particles, a measurement of the momentum from the tracking system 
can be combined with a measurement of the velocity to determine the mass from 
the relation $\ptot = \mass\beta\gamma c$.
As $\gamma$ is a function only of $\beta$, as in 
\cref{eqn:intro:sm:lorentz_factor}, and $\beta = v/c$, only one of $\gamma$ and 
$\beta$ needs to be measured.

One method for measuring $\beta$ is to measure the time taken for the particle 
to travel between two detectors.
The measured time difference between detection times $t_{1}$ and $t_{2}$ is 
equal to the distance between the two detectors $L$ divided by the particle 
velocity
\begin{equation}
  \delta t = t_{1} - t_{2} = \frac{L}{v}
           = \frac{L}{\beta c}.
\end{equation}
For a given momentum \ptot, particles with different masses will have different 
velocities, and hence different $\beta$ factors.
In the ultra-relativistic limit, where $pc \gg mc^{2}$, the difference between 
times for two particles with masses $m_{a}$ and $m_{b}$ will be
\begin{equation}
  \Delta t = \delta t_{a} - \delta t_{b}
           \approx \frac{Lc}{2p^{2}}(m_{a}^{2} - m_{b}^{2}),
\end{equation}
The utility of such a technique reduces as the particle momentum increases, as 
a very high precision is required on the time difference measurement in order 
to be able to distinguish between different particle species.

Another \ac{PID} technique is to exploit the phenomenon of Cherenkov radiation, 
where charged particles emit light when they traverse a medium at faster than 
the phase velocity of light in that medium.
The phase velocity of light, or just `the speed of light', in a medium with a 
refractive index $n$ is
\begin{equation}
  v_{\text{Light}} = \frac{c}{n}.
\end{equation}
A front of Cherenkov light is emitted at an angle \cherenkovangle\ to the 
trajectory of the particle, and this angle is related to the speed of light in 
the medium and the velocity magnitude of the particle
\begin{equation}
  \cos{\cherenkovangle} = \frac{v_{\text{Light}}}{v_{\text{Particle}}}
                        = \frac{c}{n\beta c}
                        = \frac{1}{n\beta},
\end{equation}
as illustrated in \cref{fig:intro:lhcb:cherenkov}.
Given that the medium is chosen by the experimenter, so that $n$ is known, 
$\beta$ can be measured by measuring the angle of the Cherenkov light.
As in the time-of-flight method, different particle species with the same 
measured momentum will have different values of $\beta$, and in this case 
different angles \cherenkovangle.
TODO: talk about the precision, as in the time-of-flight example.

% The main problem in a particle physics experiment is that one does not know 
% every property about what is produced in a collision.
% The design of a particle detector tries to maximise the information one can 
% obtain about the collision by exploiting the interactions of the outgoing 
% particles with the material of the detector.
% Given the information made by the detector, one can then try to infer 
% additional information about the event, including information on particles that 
% did not interact with the detector directly.

% The decay of the \PBs meson to two muons, \decay{\PBs}{\Pmuon\APmuon}, is a 
% good example of the types of processes that \lhcb was designed to detect.
% Due to the nature of the proton-proton interactions, the particles produced 
% within the detector are highly boosted with respect to the laboratory, having 
% velocities close to the speed of light.
% This means the lifetime of the particles in the lab frame are much higher than 
% if they were produced as rest.
% The \ABs meson has a lifetime at rest of \SI{2}{\pico\second}, and so can 
% travel several millimetres in the lab.
% Muons have a lifetime at rest of \SI{2}{\micr\second}, and so can travel much 
% further, tens of metres.
% This is one way to distinguish between particles.
% Another way is through the particle's electric charge.
% The \ABs meson is neutral, whereas the muon is charged, and so will ionise the 
% material of the detector much more readily.
% The ionisation produced in the material can be converted to a electric impulse 
% in a circuit, and several such impulses

% The particles can be categorised in several ways: firstly, the \PBs meson is 
% neutral, whereas the muons are charged; and secondly the \PBs meson is very 
% unstable, with a life time of \SI{2}{\pico\second}, whereas the muons are 
% stable.

The techniques which have been discussed for measuring particle momentum, 
energy, and identity are general.
The technologies used to construct an experiment depend on the physics that 
experiment intends to measure.

\begin{figure}
  \centering
  \begin{tikzpicture}[
  x={(0.866cm, -0.5cm)}, y={(0.866cm, 0.5cm)}, z={(0cm, 1cm)},
  scale=0.9,
  inner sep=0pt, outer sep=2pt,
  axis/.style={thick,->},
  station/.style={fill=black!40!white, opacity=0.3},
  particle/.style={->, dashed},
  hit/.style={star,star points=7,draw=black!70,fill=red!70,inner sep=0pt,minimum size=0.2cm}
  ]

  % Coordinate axes
  \coordinate (O) at (0, 0, 0);
  \draw[axis] (O) -- +(14, 0, 0) node [right] {$z$};
  \draw[axis] (O) -- +(0, 2.5, 0) node [right] {$x$};
  \draw[axis] (O) -- +(0, 0, 2) node [left] {$y$};

  % Upstream tracking stations
  \filldraw[station] (1, -2, -1.5) -- (1, -2, 1.5) -- (1, 2, 1.5) -- (1, 2, -1.5) -- (1, -2, -1.5);
  \filldraw[station] (2, -2, -1.5) -- (2, -2, 1.5) -- (2, 2, 1.5) -- (2, 2, -1.5) -- (2, -2, -1.5);
  \filldraw[station] (3, -2, -1.5) -- (3, -2, 1.5) -- (3, 2, 1.5) -- (3, 2, -1.5) -- (3, -2, -1.5);
  % Downstream tracking stations
  \filldraw[station] (8, -2, -1.5) -- (8, -2, 1.5) -- (8, 2, 1.5) -- (8, 2, -1.5) -- (8, -2, -1.5);
  \filldraw[station] (9, -2, -1.5) -- (9, -2, 1.5) -- (9, 2, 1.5) -- (9, 2, -1.5) -- (9, -2, -1.5);
  \filldraw[station] (10, -2, -1.5) -- (10, -2, 1.5) -- (10, 2, 1.5) -- (10, 2, -1.5) -- (10, -2, -1.5) node [below, sloped, near end] {\tiny{Tracking station}};

  % Muons
  \draw[particle] plot [smooth] coordinates {(O) (5.5, 0, 1.5) (13, 0, 0.5)} node [right] {$\mu^{-}$};
  \draw[particle] plot [smooth] coordinates {(O) (4.5, 0, -1.5) (13, 0, -1.0)} node [right] {$\mu^{+}$};

  % Hits
  % Negative muon at upstream stations
  \node[hit] at (1, 0, 0.3) {};
  \node[hit] at (1.1, 0, 0.4) {};
  \node[hit] at (3, 0, 0.9) {};
  \node[hit] at (3, 0, 1.0) {};
  \node[hit] at (4.5, 0, 1.5) {};
  \node[hit] at (4.3, 0, 1.3) {};
  \node[hit] at (4.4, 0, 1.3) {};
  % Negative muon at downstream stations
  \node[hit] at (8, 0, 1.2) {};
  \node[hit] at (8.1, 0, 1.4) {};
  % \node[hit] at (9.2, 0, 1.1) {};
  \node[hit] at (9.3, 0, 1.0) {};
  \node[hit] at (11.5, 0, 0.8) {};
  \node[hit] at (11.3, 0, 0.8) {};
  \node[hit] at (11.4, 0, 0.9) {};

  % Magnetic field vector
  \draw[thick,->] (5, 2, 1.5) -- (5, 2, 1.0) node [anchor=west] {$\vec{B}$ field} -- (5, 2, 0.5);
\end{tikzpicture}

  \caption{%
    Illustration of particle tracking, showing two muons traversing through a 
    series of tracking stations, with a magnetic field in the negative $y$ 
    direction, bending charged particles in the $xz$ plane.
    Hits made by the negatively charged muon are shown as red points.
    The scatter the hits around the true particle trajectory signifies the 
    inherent uncertainty in the experimental measurement.
  }
  \label{fig:intro:lhcb:tracking}
\end{figure}

% \begin{figure}
%   \centering
%   % Draw perpendicular markers at line intersections
% http://tex.stackexchange.com/a/21759/45857
\tikzset{
  right angle quadrant/.code={
    \pgfmathsetmacro\quadranta{{1,1,-1,-1}[#1-1]}     % Arrays for selecting quadrant
    \pgfmathsetmacro\quadrantb{{1,-1,-1,1}[#1-1]}},
  right angle quadrant=1, % Make sure it is set, even if not called explicitly
  right angle length/.code={\def\rightanglelength{#1}},   % Length of symbol
  right angle length=2ex, % Make sure it is set...
  right angle symbol/.style n args={3}{
    insert path={
      let \p0 = ($(#1)!(#3)!(#2)$) in     % Intersection
      let \p1 = ($(\p0)!\quadranta*\rightanglelength!(#3)$), % Point on base line
      \p2 = ($(\p0)!\quadrantb*\rightanglelength!(#2)$) in % Point on perpendicular line
      let \p3 = ($(\p1)+(\p2)-(\p0)$) in  % Corner point of symbol
      (\p1) -- (\p3) -- (\p2)
    }
  }
}
\begin{tikzpicture}[
  x=2cm,
  y=2cm,
  axis/.style={very thick,->,gray},
  beam/.style={very thick,->,gray},
  scale=1.5,
  thick
  ]
  % Origin
  \coordinate (O) at (0, 0);
  % B decay vertex
  \coordinate (Bvtx) at (1, 0.5);
  % Offset of muon arrows, with respect to Bvtx
  \coordinate(mumoffset) at (0.3, 0.4);
  \coordinate(mupoffset) at (0.6, -0.2);
  % Compute the coordinates of the ends of the muon lines
  \coordinate (mum) at ($(Bvtx) + (mumoffset)$);
  \coordinate (mup) at ($(Bvtx) + (mupoffset)$);

  % Coordinate axes
  \draw[axis] (-0.6, -0.5) -- +(0.2, 0) node [below] {$z$};
  \draw[axis] (-0.6, -0.5) -- +(0, 0.2) node [left] {$y$};

  \draw[beam] (-1.0, 0) -- (-0.1, 0) node [below, at start] {$p$};
  \draw[beam] (1.5, 0) -- (0.1, 0) node [below, at start] {$p$};

  % D meson
  \draw[dashed, color=gray, text=black] (O) -- (Bvtx) node [above, pos=0.5] {\PDz};
  % Negative child
  \draw[->] (Bvtx) -- (mum) node [above] {\PKminus};
  \draw[dotted] (Bvtx) -- ($(Bvtx) - 3*(mumoffset)$) node [name=mumend] {};
  % Positive child
  \draw[->] (Bvtx) -- (mup) node [below] {\Ppiplus};
  \draw[dotted] (Bvtx) -- ($(Bvtx) - 2*(mupoffset)$) node [name=mupend] {};

  % Primary vertex
  \node[star,star points=10,draw=orange!50,fill=orange!20,inner sep=0pt,minimum size=0.4cm] at (O) {};

  % Negative muon IP
  \draw[right angle length=1mm, right angle symbol={Bvtx}{mumend}{O}] ($(Bvtx)!(O)!(mumend)$) -- (O) node [pos=0.2, left] {$\text{IP}_{\PKminus}$};
  % Positive muon IP
  \draw[right angle length=1mm, right angle symbol={Bvtx}{mupend}{O}] ($(Bvtx)!(O)!(mupend)$) -- (O) node [midway, left] {$\text{IP}_{\Ppiplus}$};
\end{tikzpicture}

%   \caption{%
%     Illustration of vertexing, showing a \PBs meson decaying in flight to two 
%     muons.
%     The minimum transverse distance the muon tracks make when extrapolated back 
%     towards the primary proton-proton vertex, the \acf{IP}, is shown.
%   }
%   \label{fig:intro:lhcb:vertexing}
% \end{figure}

\begin{figure}
  \centering
  \begin{tikzpicture}
  \coordinate (O) at (0, 0);
  \pgfmathsetmacro{\ringbeginx}{5};
  \pgfmathsetmacro{\ringradius}{2};

  % Momentum vector
  \draw[-latex] ($(O) - (0.5, 0)$) -- (O) -- (\ringbeginx, 0) -- +(1, 0) node [anchor=west] {\ptot};

  % Cherenkov light
  \draw[dashed] (O) -- +(\ringbeginx, \ringradius) node [pos=0.5, anchor=south east] {$\frac{c}{n}t$};
  \draw[dashed] (O) -- +(\ringbeginx, -\ringradius);
  \draw (1, 0) arc (0:22:1);
  \node at (10:1.25)  {{\footnotesize $\theta_{c}$}};

  % Projection of light on to detection surface, forming a ring
  \draw[dotted] (\ringbeginx, 0) ellipse (0.1cm and \ringradius cm);
  % Ring radius
  \draw[<->] ($(\ringbeginx, 0) + (0.3, 0)$) -- ($(\ringbeginx, -\ringradius) + (0.3, 0)$) node [pos=0.5, anchor=west] {$R$};

  % Measurement of particle flight distance
  \draw[<->]  ($(O) + (0, -\ringradius) - (0, 0.2)$) -- ($(\ringbeginx, -\ringradius) - (0, 0.2)$) node [pos=0.5, anchor=south] {$\beta ct$};
\end{tikzpicture}

  \caption{%
    Illustration of Cherenkov light being emitted from a particle travelling 
    with momentum $p$ and velocity $\beta$ through a medium with refractive 
    index $n$.
    The radiation is emitted at a angle \cherenkovangle\ to the trajectory of 
    the particle, and forms a ring of radius $R$ when projected onto a plane 
    transverse to the particle momentum direction.
  }
  \label{fig:intro:lhcb:cherenkov}
\end{figure}

\section{Physics goals}
\label{chap:intro:lhcb:physics}

\section{Detector}
\label{chap:intro:lhcb:detector}

\subsection{Tracking}
\label{chap:intro:lhcb:detector:tracking}

\subsection{Particle identification}
\label{chap:intro:lhcb:detector:pid}

\subsection{Event selection}
\label{chap:intro:lhcb:detector:trigger}

\subsection{Upgrades in \acl{LS1}}
\label{chap:intro:lhcb:detector:upgrades}
