\chapter{Overview}
\label{chap:intro:overview}

The Large Hadron Collider is particle accelerator at CERN, a particle physics 
laboratory, in Geneva, Switzerland.
It is a ring \SI{27}{\kilo\metre} in circumference designed to accelerate two 
oppositely-circulating beams of protons up to energies of \SI{8}{\TeV}.
The beams collide into each other at four points around the ring, at each of 
which a particle detector records the resulting showers of particles.

Two of these detectors, ATLAS and CMS, are designed to study a wide range of 
phenomena, including searches for particles not seen before the \ac{LHC} such 
as the Higgs boson, while ALICE is designed to study the quark-gluon plasma, 
the primary state of matter that existed in the early universe.
This thesis is concerned with the LHCb detector, which is designed to measure 
the properties of hadrons containing charm and beauty quarks, the two heaviest 
quarks that can form bound states.

The observed imbalance of matter and anti-matter in the universe cannot be 
explained by established theoretical models, and the physics of heavy flavour, 
interactions involving charm and beauty quarks, can provide some insight in to 
possible causes.
The properties and decays of beauty hadrons have been observed to be different 
between matter and anti-matter, and similar behaviour is expected in charm 
hadrons, albeit with a smaller magnitude.
The high energy of the \ac{LHC} beams produces large samples of charm and 
beauty hadrons, allowing for precise measurements of their properties to be 
made.

This thesis will cover two different topics: physics analysis, and detector 
monitoring.
The former involves using data collected by the detector to measure specific 
physical quantities.
The first analysis presented here is a measurement of the rate of production of 
mesons containing charm quarks, performed using data taken at a proton-proton 
centre-of-mass energy of \runtwocom, presented in \cref{chap:prod}.
The second analysis is a measurement of the relative rate of the decays of 
charm and anti-charm baryons, presented in \cref{chap:cpv}.
\cref{chap:velo} describes the motivation for and implementation of a new 
software framework for monitoring the health and performance of the Vertex 
Locator sub-detector.

The design and operation of the LHCb detector plays an important part in all 
that follows, and so it shall be described in detail in \cref{chap:intro:lhcb}.
The physical context in which LHCb lives, the \ac{LHC}, will be presented 
in \cref{chap:intro:lhc}, whilst the ideological context, the \ac{SM}, will be 
presented in \cref{chap:intro:sm}.
