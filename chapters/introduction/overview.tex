\chapter{Overview}
\label{chap:intro:overview}

The Large Hadron Collider is particle accelerator at CERN, a particle physics 
laboratory, in Geneva, Switzerland.
It is a ring \SI{27}{\kilo\metre} in circumference designed to accelerate two 
oppositely-circulating beams of protons up to energies of \SI{8}{\TeV}.
The beams collide into each other at four points around the ring, at each of 
which a particle detector records the resulting showers of particles.

Two of these detectors, ATLAS and CMS, are designed to study a wide range of 
phenomena, including searches for particles not seen before the \ac{LHC}, such 
as the Higgs boson, while ALICE is designed to study the quark-gluon plasma, 
the primary state of matter that existed in the early universe.
This thesis focuses on the LHCb detector, which is designed to measure the 
properties of hadrons containing charm and beauty quarks, the two heaviest 
quarks that can form bound states.

The observed imbalance of matter and anti-matter in the universe cannot be 
explained by established theoretical models, and the physics of heavy flavour, 
interactions involving charm and beauty quarks, can provide some insight into 
possible causes.
The properties and decays of beauty hadrons have been observed to be different 
between matter and 
anti-matter~\cite{Aubert:2001nu,Abe:2001xe,Aaij:2012kz,Aaij:2013iua,Aaij:2016cla}, 
and similar behaviour is expected in charm hadrons, albeit with a smaller 
magnitude~\cite{Grossman:2006jg}.
The high energy of the \ac{LHC} beams produces large samples of charm and 
beauty hadrons~\cite{LHCb-PAPER-2012-041,LHCb-PAPER-2013-004}, allowing for 
precise measurements of their properties to be made.

This thesis will cover the two different analyses, each of which involves using 
data collected by the detector to measure specific physical quantities.
The first analysis presented is a measurement of the rate of production of 
mesons containing charm quarks, performed using data taken at a proton-proton 
centre-of-mass energy of \runtwocom, presented in \cref{chap:prod}.
The second analysis is a measurement of the relative rate of the decays of 
charm and anti-charm baryons, presented in \cref{chap:cpv}.
As both analyses share similar foundations, a general introduction is given in 
\cref{chap:intro}.
The \acl{SM}, the mathematical framework which is able to describe most of what 
is known about fundamental particles and their interactions, will be presented 
in \cref{chap:intro:sm}.
A description of the \acl{LHC} and its operation will be presented in 
\cref{chap:intro:lhc}, before a description of the physics goals and design of 
the \lhcb\ experiment.
