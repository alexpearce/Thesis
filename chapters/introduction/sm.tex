\chapter{The \acl{SM}}
\label{chap:intro:sm}

The \acf{SM} is a quantum field theory that describes the interactions of three 
of the four fundamental forces of nature, electromagnetism, the weak force, and 
the strong force, with a set of fundamental particles.
The predictions made by the \ac{SM} are legion, and so far no experimental 
results have provided conclusive proof that these predictions are incorrect.
However, astronomical observations indicate that, at the very least, the 
\ac{SM} is not a \emph{complete} understanding of the universe, in that it does 
not describe the observed matter-antimatter asymmetry, nor the apparent 
presence of dark matter.
The purpose of the \ac{LHC} is to provide experiments with enough data to be 
able to find minute deviations from the predictions on the \ac{SM} that would 
point to alternate theories of nature.

In this \namecref{chap:intro:sm}, the forces and particles described by the 
\ac{SM} shall be summarised, ending with a particular emphasis on the 
phenomenology of the weak force, which the \lhcb\ detector is optimised to 
study.

\subsection{Particles and forces}

The \ac{SM} can be described equivalently by the interactions between fields or 
between particles, or by a mixture of the two.
In the particle formulation, forces are generated by the emission and 
absorption of gauge bosons.
The \ac{EM} force acts between particles with a non-zero electric charge via 
exchange of photons.
The electric charge is an example of the conserved charge that is associated 
with a field, the \ac{EM} field in this case.
Interactions with a field preserve the total associated quantum charge of the 
system, and the field does not interact with particles with a charge of zero.

The strong force acts on particles with a non-zero colour charge via the 
exchange of gluons, with quarks and gluons being the only particles that have a 
non-zero colour charge.
Gluons being charged under the force they mediate significantly complicates the 
theory of strong interactions, called \ac{QCD}, in comparison with \ac{EM} 
interactions.
At low energies, this self interaction leads to the phenomenon of confinement, 
whereby colour-charged objects cannot be observed experimentally, as the 
attractive force between two colour-charged object is constant with increasing 
distance, requiring an infinite amount of energy to separate them completely.
Conversely, at higher energies, these particles become asymptotically free, 
eventually forming the quark-gluon plasma.
The six quarks, up, down, charm, strange, top, and bottom (\Pup, \Pdown, 
\Pcharm, \Pstrange, \Ptop, and \Pbottom, generically \Pquark), are only 
observed in colourless bound states of two or more quarks called hadrons.
Mesons are hadrons made of two quarks, and baryons are hadrons made of three 
quarks.
The family of baryons includes the proton and the neutron, made up of 
$\Pup\Pup\Pdown$ and $\Pup\Pdown\Pdown$ quarks respectively, which together 
make up the bulk of the mass of atoms.
The binding of protons and neutrons together in an atomic nucleus is mediated 
indirectly by the strong force.

The last of the three forces described by the \ac{SM} is the weak force, which 
interacts with particles with a non-zero weak isospin \wisospin.
The carriers of the weak force are the two charged \PWpm bosons and the neutral 
\PZ boson.
The quarks can be categorised by their weak isospin values into the up-type 
quarks with $\wisospin = \sfrac{1}{2}$, up, charm, and top, and the down-type 
quarks with $\wisospin = -\sfrac{1}{2}$, down, strange, and bottom.
Likewise, the leptons, another family of fundamental particles, can also be 
categorised by their weak isospin values into the electron-like particles with 
$\wisospin = -\sfrac{1}{2}$, electron, muon, and tau particle, and the 
neutrino-like particles with $\wisospin = \sfrac{1}{2}$, electron neutrino, 
muon neutrino, and tau neutrino (denoted \Pelectron, \Pmuon, \Ptau, \Pnue, 
\Pnum, and \Pnut).
The quarks and leptons are often written as \emph{doublets} of particles, pairs 
with opposite values of the weak isospin
\begin{equation}
  \begin{pmatrix}\Pup\\\Pdown\end{pmatrix},\,
  \begin{pmatrix}\Pcharm\\\Pstrange\end{pmatrix},\,
  \begin{pmatrix}\Ptop\\\Pbottom\end{pmatrix}
  \text{\ and\ }
  \begin{pmatrix}\Pnue\\\Pelectron\end{pmatrix},\,
  \begin{pmatrix}\Pnum\\\Pmuon\end{pmatrix},\,
  \begin{pmatrix}\Pnut\\\Ptau\end{pmatrix}.
  \label{intro:sm:weak_doublets}
\end{equation}
Particles within these doublets can interact via the exchange of \PWpm bosons, 
and it is this mechanism that permits the radioactive decay of atoms, wherein a 
neutron decays into a proton, an electron, and an electron neutrino by the 
decay of one of the down quarks in the neutron.
This process is illustrated in \cref{fig:intro:sm:neutron_decay} as a Feynman 
diagram.

\begin{figure}
  \centering
  % \begin{tikzpicture}
%   \begin{feynman}
%     \vertex (a);
%     \vertex [left=0.1cm of a] (a1);
%     \vertex [left=0.1cm of a1] (a11);
%     \vertex [right of=a] (b);
%     \vertex [above of=b] (c);
%     \vertex [below left of=a] (i1);
%     \vertex [left=0.1cm of i1] (i11);
%     \vertex [left=0.1cm of i11] (i111);
%     \vertex [above left of=a] (i2);
%     \vertex [left=0.1cm of i2] (i21);
%     \vertex [left=0.1cm of i21] (i211);
%     \vertex [left=0.3cm of c] (f1) {\(\bar{\nu}_{e}\)};
%     \vertex [right=0.3cm of c] (f2) {\(e^{-}\)};
%     \diagram*[small] {%
%       (i1) -- [fermion, edge label'=\(d\)] (a) -- [fermion, edge label'=\(u\)] (i2),
%       (i11) -- (a1) -- (i21),
%       (i111) -- (a11) -- (i211),
%       (a) -- [boson, edge label=\(W^{-}\)] (b),
%       (f1) -- [fermion] (b) -- [fermion] (f2),
%     };
%   \end{feynman}
% \end{tikzpicture}
\feynmandiagram [layered layout, horizontal=a to b] {%
  a [particle=\(\Pdown\)] -- [fermion] b -- [fermion] f1 [particle=\(\Pup\)],
  b -- [boson, edge label'=\(\PWm\)] c,
  c -- [anti fermion] f2 [particle=\(\APnue\)],
  c -- [fermion] f3 [particle=\(\Pelectron\)],
};

  \caption{%
    Feynman diagram for the decay of a down quark into an up quark, an 
    electron, and an electron neutrino, \decay{\Pdown}{\Pup\Pelectron\APnue}.
    The process conserves charge, lepton number, and weak isospin.
  }
  \label{fig:intro:sm:neutron_decay}
\end{figure}

% TODO higgs here?

Most particles eventually decay.
This is very often via a process mediated by the weak force, as in 
\cref{fig:intro:sm:neutron_decay}, but can also be mediated by the strong 
force.
Different particles take different lengths of time to decay, on average, to 
decay than others, and this is parameterised as the lifetime \lifetime.
The probability $P$ that a given particle will decay in a time $t$ after it has 
been created is
\begin{equation}
  P(t) = \frac{1}{\lifetime}e^{-\frac{t}{\lifetime}}.
  \label{eqn:intro:sm:lifetime}
\end{equation}
The time $t$ here is measured in the rest frame of the particle.
If the particle is moving at a velocity $v$ in some other frame, such as in a 
laboratory, the lifetime of the particle in the lab changes due to time 
dilation
\begin{equation}
  \lifetimelab = \gamma\lifetime,
  \label{eqn:intro:sm:lifetime_lab}
\end{equation}
where $\gamma$ is the Lorentz factor from special relativity
\begin{equation}
  \gamma = \frac{1}{\sqrt{1 - \frac{v^{2}}{c^{2}}}},
         = \frac{1}{\sqrt{1 - \beta^{2}}},
  \label{eqn:intro:sm:lorentz_factor}
\end{equation}
where $c$ is the speed of light in a vacuum.

Any particles, fundamental and composite, will eventually decay if such a 
process is not forbidden.
A process is forbidden if some conserved quantum number, such as the electric 
charge, would not be conserved.
Other conserved quantum numbers include the baryon number, which is 1 for 
baryons and $-1$ for anti-baryons, and the lepton number, which is 1 for 
leptons and $-1$ for anti-leptons.
By conserving such quantum numbers, and by the conservation of energy, the only 
particles predicted to be stable, having lifetimes greater than the age of the 
universe, are the photon, the electron, the neutrinos, and the proton.
Particle decays cascade down until the final state, the total set of stable 
particles, contains these particles.
As some particles have lifetimes to small to be detected experimentally, the 
decay products with longer lifetimes can be studied instead, and properties 
about the parent can be inferred from the properties of the children.

Each of the fundamental particles described so far can each have an associated 
antiparticle, which is nearly identical to the particle but has the opposite 
electric charge.
The neutral photon, gluon, and \PZ boson are each their own antiparticles, and 
the \PWp and \PWm bosons are each other's antiparticles.
For the other particles, the antiparticle is denoted with a bar, such as the 
anti-up quark \APup, or with the opposite charge, such as the anti-electron 
\APelectron (also called the positron).
Although particles are similar to their antiparticles in almost all measurable 
ways, the \ac{SM} does treat matter and antimatter differently, and in this way 
is not symmetric between them.

\subsection{Symmetries}

There are parts of the mathematical formulation of the \ac{SM} that can be 
altered without the physically observable parameters changing.
Such invariance is called a \emph{symmetry}.
Symmetries give rise to conservation laws, and vice versa, by Noether's 
theorem.
The existence of symmetries in a physical system is often first postulated 
simply by intuition, for example it may seem reasonable that a physical process 
behaves the same if the system is translated in space and in time.
This corresponds to two symmetries, and these give rise to the conservation of 
momentum and the conservation of energy respectively, both of which are 
predicted by the \ac{SM}.

In addition to spatial symmetries, such as under translation transformations, 
the \ac{SM} has several internal symmetries, which correspond to invariance 
under some interaction between particles.
The conservation of electric charge, baryon number, and lepton number are 
examples of internal symmetries.
There are some transformations that the \ac{SM} is not invariant under.
Two examples are the charge conjugation transformation \Ctransform\ and the 
parity transformation \Ptransform.
The former changes the sign of all quantum charges in the system, changing a 
particle in to its antiparticle.
The latter changes the sign of all spatial coordinates
\begin{equation}
  P: \begin{pmatrix}x\\y\\z\end{pmatrix}
     \mapsto \begin{pmatrix}-x\\-y\\-z\end{pmatrix}.
  \label{eqn:intro:sm:parity}
\end{equation}

It was believed that the \Ctransform\ and \Ptransform\ transformations were 
conserved in all interactions, but it was then observed experimentally that 
this was not the case; the symmetries are \emph{violated} in nature.
Interactions with the electromagnetic and the strong force do respect these 
symmetries, but interactions with the weak force do not.
In an attempt to restore some sort of order, it was postulated that the weak 
force would be symmetric under the combination of the two transformations, \CP, 
but this was also found to be violated experimentally.

The violation of the \Ctransform, \Ptransform, and \CP\ symmetries was 
incorporated into the \ac{SM}, and further experimental evidence of these 
phenomena has agreed with the \ac{SM} predictions.
However, the degree to which the interactions and production of matter and 
antimatter differ in the \ac{SM} is small, to the extent that it is an open 
question as to why the visible universe is dominated by matter.
The study of \CP\ violation in and beyond the \ac{SM} is a key goal of the 
\lhcb\ experiment, and will be discussed in more detail in 
\cref{chap:cpv:theory}.
