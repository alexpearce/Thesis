\chapter{The \acl{SM}}
\label{chap:intro:sm}

The \acf{SM} is a quantum field theory that describes the interactions of three 
of the four fundamental forces of nature: electromagnetism, the weak force, and 
the strong force.
These forces interact with and between a set of fundamental particles, 
particles which have no known substructure.
The behaviour of these particles can be observed or inferred experimentally to 
deduce the nature of the forces, such how their strength changes with distance.
This \namecref{chap:intro:sm} shall enumerate and describe these particles and 
forces, before discussing why additional theories are required and how they can 
be verified.

\subsection{Particles and forces}

The \ac{SM} can be described equivalently by the interactions between fields or 
between particles, or by a mixture of the two.
In the particle formulation, forces are generated by the emission and 
absorption of gauge bosons.
The \ac{EM} force acts between particles with a non-zero electric charge via 
exchange of photons.
The electric charge is an example of the conserved charge that is associated 
with a field, the \ac{EM} field in this case.
Interactions with a field preserve the total associated quantum charge of the 
system, and the field does not interact with particles with a charge of zero.

The strong force acts on particles with a non-zero colour charge via the 
exchange of gluons, with quarks and gluons being the only particles that have a 
non-zero colour charge.
Gluons being charged under the force they mediate significantly complicates the 
theory of strong interactions, called \ac{QCD}, in comparison with \ac{EM} 
interactions.
At low energies, this self interaction leads to the phenomenon of confinement, 
whereby colour-charged objects cannot be observed experimentally, as the 
attractive force between two colour-charged object is constant with increasing 
distance, requiring an infinite amount of energy to separate them completely.
Conversely, at higher energies, these particles become asymptotically free, 
eventually forming the quark-gluon plasma.
The six quarks, up, down, charm, strange, top, and bottom (\Pup, \Pdown, 
\Pcharm, \Pstrange, \Ptop, and \Pbottom, generically \Pquark), are only 
observed in colourless bound states of two or more quarks called hadrons.
Mesons are hadrons made of two quarks, and baryons are hadrons made of three 
quarks.
The family of baryons includes the proton and the neutron, made up of 
$\Pup\Pup\Pdown$ and $\Pup\Pdown\Pdown$ quarks respectively, which together 
make up the bulk of the mass of atoms.
The binding of protons and neutrons together in an atomic nucleus is mediated 
indirectly by the strong force.

The last of the three forces described by the \ac{SM} is the weak force, which 
interacts with particles with a non-zero weak isospin \wisospin.
The carriers of the weak force are the two charged \PWpm bosons and the neutral 
\PZ boson.
The quarks can be categorised by their weak isospin values into the up-type 
quarks with $\wisospin = \sfrac{1}{2}$, up, charm, and top, and the down-type 
quarks with $\wisospin = -\sfrac{1}{2}$, down, strange, and bottom.
Likewise, the leptons, another family of fundamental particles, can also be 
categorised by their weak isospin values into the electron-like particles with 
$\wisospin = -\sfrac{1}{2}$, electron, muon, and tau particle, and the 
neutrino-like particles with $\wisospin = \sfrac{1}{2}$, electron neutrino, 
muon neutrino, and tau neutrino (denoted \Pelectron, \Pmuon, \Ptau, \Pnue, 
\Pnum, and \Pnut).
The quarks and leptons are often written as \emph{doublets} of particles, pairs 
with opposite values of the weak isospin
\begin{equation}
  \begin{pmatrix}\Pup\\\Pdown\end{pmatrix},\,
  \begin{pmatrix}\Pcharm\\\Pstrange\end{pmatrix},\,
  \begin{pmatrix}\Ptop\\\Pbottom\end{pmatrix}
  \text{\ and\ }
  \begin{pmatrix}\Pnue\\\Pelectron\end{pmatrix},\,
  \begin{pmatrix}\Pnum\\\Pmuon\end{pmatrix},\,
  \begin{pmatrix}\Pnut\\\Ptau\end{pmatrix}.
  \label{intro:sm:weak_doublets}
\end{equation}
Particles within these doublets can interact via the exchange of \PWpm bosons, 
and it is this mechanism that permits the radioactive decay of atoms, wherein a 
neutron decays into a proton, an electron, and an electron neutrino by the 
decay of one of the down quarks in the neutron.
This process is illustrated in \cref{fig:intro:sm:neutron_decay} as a Feynman 
diagram.

\begin{figure}
  \centering
  % \begin{tikzpicture}
%   \begin{feynman}
%     \vertex (a);
%     \vertex [left=0.1cm of a] (a1);
%     \vertex [left=0.1cm of a1] (a11);
%     \vertex [right of=a] (b);
%     \vertex [above of=b] (c);
%     \vertex [below left of=a] (i1);
%     \vertex [left=0.1cm of i1] (i11);
%     \vertex [left=0.1cm of i11] (i111);
%     \vertex [above left of=a] (i2);
%     \vertex [left=0.1cm of i2] (i21);
%     \vertex [left=0.1cm of i21] (i211);
%     \vertex [left=0.3cm of c] (f1) {\(\bar{\nu}_{e}\)};
%     \vertex [right=0.3cm of c] (f2) {\(e^{-}\)};
%     \diagram*[small] {%
%       (i1) -- [fermion, edge label'=\(d\)] (a) -- [fermion, edge label'=\(u\)] (i2),
%       (i11) -- (a1) -- (i21),
%       (i111) -- (a11) -- (i211),
%       (a) -- [boson, edge label=\(W^{-}\)] (b),
%       (f1) -- [fermion] (b) -- [fermion] (f2),
%     };
%   \end{feynman}
% \end{tikzpicture}
\feynmandiagram [layered layout, horizontal=a to b] {%
  a [particle=\(\Pdown\)] -- [fermion] b -- [fermion] f1 [particle=\(\Pup\)],
  b -- [boson, edge label'=\(\PWm\)] c,
  c -- [anti fermion] f2 [particle=\(\APnue\)],
  c -- [fermion] f3 [particle=\(\Pelectron\)],
};

  \caption{%
    Feynman diagram for the decay of a down quark into an up quark, an 
    electron, and an electron neutrino, \decay{\Pdown}{\Pup\Pelectron\APnue}.
    The process conserves charge, lepton number, and weak isospin.
  }
  \label{fig:intro:sm:neutron_decay}
\end{figure}

% TODO higgs here?

Most particles eventually decay.
This is very often via a process mediated by the weak force, as in 
\cref{fig:intro:sm:neutron_decay}, but can also be mediated by the strong 
force.
Different particles take different lengths of time to decay, on average, to 
decay than others, and this is parameterised as the lifetime \lifetime.
The probability $P$ that a given particle will decay in a time $t$ after it has 
been created is
\begin{equation}
  P(t) = \frac{1}{\lifetime}e^{-\frac{t}{\lifetime}}.
  \label{eqn:intro:sm:lifetime}
\end{equation}
The time $t$ here is measured in the rest frame of the particle.
If the particle is moving at a velocity $v$ in some other frame, such as in a 
laboratory, the lifetime of the particle in the lab changes due to time 
dilation
\begin{equation}
  \lifetimelab = \gamma\lifetime,
  \label{eqn:intro:sm:lifetime_lab}
\end{equation}
where $\gamma$ is the Lorentz factor from special relativity
\begin{equation}
  \gamma = \frac{1}{\sqrt{1 - \frac{v^{2}}{c^{2}}}},
         = \frac{1}{\sqrt{1 - \beta^{2}}},
  \label{eqn:intro:sm:lorentz_factor}
\end{equation}
where $c$ is the speed of light in a vacuum.

Any particles, fundamental and composite, will eventually decay if such a 
process is not forbidden.
A process is forbidden if some conserved quantum number, such as the electric 
charge, would not be conserved.
Other conserved quantum numbers include the baryon number, which is 1 for 
baryons and $-1$ for anti-baryons, and the lepton number, which is 1 for 
leptons and $-1$ for anti-leptons.
By conserving such quantum numbers, and by the conservation of energy, the only 
particles predicted to be stable, having lifetimes greater than the age of the 
universe, are the photon, the electron, the neutrinos, and the proton.
Particle decays cascade down until the final state, the total set of stable 
particles, contains these particles.
As some particles have lifetimes to small to be detected experimentally, the 
decay products with longer lifetimes can be studied instead, and properties 
about the parent can be inferred from the properties of the children.

Each of the fundamental particles described so far can each have an associated 
antiparticle, which is nearly identical to the particle but has the opposite 
electric charge.
The neutral photon, gluon, and \PZ boson are each their own antiparticles, and 
the \PWp and \PWm bosons are each other's antiparticles.
For the other particles, the antiparticle is denoted with a bar, such as the 
anti-up quark \APup, or with the opposite charge, such as the anti-electron 
\APelectron (also called the positron).
Although particles are similar to their antiparticles in almost all measurable 
ways, the \ac{SM} does treat matter and antimatter differently, and in this way 
is not symmetric between them.

\subsection{Symmetries}

There are parts of the mathematical formulation of the \ac{SM} that can be 
altered without the physically observable parameters changing.
Such invariance is called a \emph{symmetry}.
Symmetries give rise to conservation laws, and vice versa, by Noether's 
theorem.
The existence of symmetries in a physical system is often first postulated 
simply by intuition, for example it may seem reasonable that a physical process 
behaves the same if the system is translated in space and in time.
This corresponds to two symmetries, and these give rise to the conservation of 
momentum and the conservation of energy respectively, both of which are 
predicted by the \ac{SM}.

In addition to spatial symmetries, such as under translation transformations, 
the \ac{SM} has several internal symmetries, which correspond to invariance 
under some interaction between particles.
The conservation of electric charge, baryon number, and lepton number are 
examples of internal symmetries.
There are some transformations that the \ac{SM} is not invariant under.
Two examples are the charge conjugation transformation \Ctransform\ and the 
parity transformation \Ptransform.
The former changes the sign of all quantum charges in the system, changing a 
particle in to its antiparticle.
The latter changes the sign of all spatial coordinates
\begin{equation}
  P: \begin{pmatrix}x\\y\\z\end{pmatrix}
     \mapsto \begin{pmatrix}-x\\-y\\-z\end{pmatrix}.
  \label{eqn:intro:sm:parity}
\end{equation}

It was believed that the \Ctransform\ and \Ptransform\ transformations were 
conserved in all interactions, but it was then observed experimentally that 
this was not the case; the symmetries are \emph{violated} in nature.
Interactions with the electromagnetic and the strong force do respect these 
symmetries, but interactions with the weak force do not.
In an attempt to restore some sort of order, it was postulated that the weak 
force would be symmetric under the combination of the two transformations, \CP, 
but this was also found to be violated experimentally.

The violation of the \Ctransform, \Ptransform, and \CP\ symmetries was 
incorporated into the \ac{SM}, and further experimental evidence of these 
phenomena has agreed with the \ac{SM} predictions.
However, the degree to which the interactions and production of matter and 
antimatter differ in the \ac{SM} is small, to the extent that it is an open 
question as to why the visible universe is dominated by matter.
The study of \CP\ violation in and beyond the \ac{SM} is a key goal of the 
\lhcb\ experiment, and will be discussed in more detail in 
\cref{chap:cpv:theory}.

% TODO could also describe how BSM models predict violations of conserved
% quantum numbers/symmetries
\section{Searching for new particles}

The \acl{SM} is able to accurately predict much of the observed phenomena in 
the universe, using only the forces and particles previously described.
Two notable phenomena the \ac{SM} does not predict are the existence of 
gravity, the fourth fundamental force of nature, and the aforementioned 
imbalance between matter and antimatter in the universe.
These shortcomings prevent a complete understanding of the physical world.
To address this, theories \acf{BSM} have been proposed.

% It is one task of high-energy physics experiments to find discrepancies with 
% the predictions of the \ac{SM}, which may provide clues as to what direction 
% theoretical physics should follow.

In addition to agreeing with existing experimental data, \ac{BSM} theories must 
make new predictions that can be tested experimentally, either using existing 
data or with new experiments.
Often, \ac{BSM} theories predict the existence of new particles, or the 
breaking of a conservation law such as the conservation of lepton number.
The principle task of the current generation of particle physics experiments at 
the \acl{LHC}, which will be described in \cref{chap:intro:lhc}, is to search 
for evidence of such phenomena.

The existence of new particles can be inferred experimentally in two ways: 
directly, where a new particle is produced and then decays into two or more 
particles, which may themselves be new predictions of the theory; and 
indirectly, where the visible constituents of a process remain the same, but 
the new particles interact internally, changing some part of the interaction, 
such as the rate at which it occurs.

\cref{fig:intro:sm:particle_production:direct} illustrates one mechanism for 
the production of a muon and an anti-muon from the annihilation of an electron 
and a positron, denoted \decay{\Pelectron\APelectron}{\Pmuon\APmuon}.
The rate at which this process occurs can be precisely computed within the 
context of the \ac{SM}, and depends on the energies of the two electrons: when 
their total energy is equal to the mass of the \PZ boson, around 
\SI{91}{\GeVcc}~\cite{PDG2014}, the rate is high relative to the rate either 
side of this value.
This is a prediction of the theory: given two electrons interacting, such as in 
a particle collider, an experiment will measure a large increase in the number 
of muon pairs when the electrons have a combined energy around \SI{91}{\GeV}.
Seeing this behaviour in an experiment would validate the prediction of theory, 
and the \PZ boson would be then be said to have been `observed'.
This is an example of a direct observation of a particle.

\cref{fig:intro:sm:particle_production:indirect} shows one mechanism for the 
decay of a \PBs meson, made of a bottom anti-quark and a strange quark, to a 
muon and an anti-muon, denoted \BsTomumu.
The intermediary particles are all the other particles in the diagram, namely 
the up-types quarks and the \PW bosons.
The \ac{SM} can also be used to predict the rate for this 
process~\cite{Bobeth:2013uxa}, which is equal to the probability that a given 
\PBs meson will decay to the $\Pmuon\APmuon$ final state, having a value around 
\num{3e-9}, or three \PBs mesons in every ten billion.
If a \ac{BSM} theory introduces new particles, they may contribute to the 
internal process in the diagram, either enhancing or reducing the rate.
An experimentalist can then count the total number of \PBs mesons observed, and 
compare that with the number of \BsTomumu\ decays.
If they see a number of those decays that is significantly larger or smaller 
than one per billion \PBs mesons, it is a sign that something beyond the 
\ac{SM} is contributing to the process.
If the experimentally observed rate is consistent with that predicted by a 
particular theory, there is evidence for the validity of that theory.

\begin{figure}
  \begin{subfigure}[b]{0.4\textwidth}
    \centering
    \feynmandiagram [large, horizontal=a to b] {%
  i1 [particle=\(\Pelectron\)] -- [fermion] a -- [fermion] i2 [particle=\(\APelectron\)],
  a -- [photon, edge label=\(\Pphoton/\PZ\)] b,
  f1 [particle=\(\APmuon\)] -- [fermion] b -- [fermion] f2 [particle=\(\Pmuon\)],
};

    \caption{Direct}
    \label{fig:intro:sm:particle_production:direct}
  \end{subfigure}
  \begin{subfigure}[b]{0.6\textwidth}
    \centering
    \begin{tikzpicture}
  \begin{feynman}
    \vertex (a1) {\APbottom};
    \vertex[right=1.5cm of a1] (a2);
    \vertex[right=1.5cm of a2] (a3);
    \vertex[right=1.5cm of a3] (a4) {\Pmuon};

    \vertex[below=4em of a1] (b1) {\Pdown};
    \vertex[right=1.5cm of b1] (b2);
    \vertex[right=1.5cm of b2] (b3);
    \vertex[right=1.5cm of b3] (b4) {\APmuon};

    \diagram* [large] {%
      {[edges=fermion]
        % Fermion line on LHS of box
        (b1) -- (b2) -- [edge label={\Pup,\Pcharm,\Ptop}] (a2) -- (a1),
        % Fermion line on RHS of box
        (b4) -- (b3) -- [edge label'={\Pup,\Pcharm,\Ptop}] (a3) -- (a4),
      },
      % Boson line at top of box
      (a2) -- [boson, edge label={\PW}] (a3),
      % Boson line at bottom of box
      (b2) -- [boson, edge label'={\PW}] (b3),
    };

    \draw [decoration={brace}, decorate] (b1.south west) -- (a1.north west)
    node [pos=0.5, left] {\PBs};
  \end{feynman}
\end{tikzpicture}

    \caption{Indirect}
    \label{fig:intro:sm:particle_production:indirect}
  \end{subfigure}
  \caption{%
    Feynman diagrams illustrating two processes through which the existence of 
    new particles may be inferred.
    Figure (\subref*{fig:intro:sm:particle_production:direct}) shows direct 
    (resonant) production of \PZ bosons, in the process 
    \decay{\Pelectron\APelectron}{\Pmuon\APmuon}.
    Figure (\subref*{fig:intro:sm:particle_production:indirect}) shows the 
    process \BsTomumu, where a new particle can enter in the `box' in the 
    centre of the diagram.
  }
  \label{fig:intro:sm:particle_production}
\end{figure}

The \PZ boson was discovered in a similar process to that depicted in 
\cref{fig:intro:sm:particle_production:direct} in 
1983~\cite{1983398,BAGNAIA1983130}, and the decay \BsTomumu\ was first observed 
in 2014.
The rate of both processes were consistent with \ac{SM} predictions.

New particles can be observed using indirect methods before collider energies 
are high enough for them to be produced directly, as was the case for the \PZ 
boson~\cite{HASERT1973138}.
This is the benefit of indirect methods, which are possible due to quantum 
mechanical principles that govern the internal interactions, in that they allow 
particles to participate in an process internally for a short amount of time, 
even if the total energy in the system is less than the mass of the particle.
% \begin{equation}
%   \Delta E \Delta t \geq \frac{\hbar}{2}.
%   \label{eqn:intro:sm:uncertainty_principle}
% \end{equation}
The existence of new particles, and more detailed properties such as their 
decay rates, can later be confirmed using direct methods.
The disadvantage of indirect methods is that it may not be conclusive what the 
nature of the new particle is, as it is interacting in somewhat subtle ways.
The indirect effect of heavy particles can be seen in 
\cref{fig:intro:sm:particle_production:indirect}, where a \PW boson enters the 
diagram despite the mass of the \PBs meson being around fifteen times 
lighter~\cite{PDG2014}.

The introduction of new particles can break the internal symmetries of the 
\ac{SM}, such as the conversation of baryon number.
For example, the decay of a proton to three muons, 
\decay{\Pproton}{\Pmuon\APmuon\Pmuon}, is forbidden in the \ac{SM} as the 
baryon number changes from one to zero.
If such a decay were to be observed experimentally, it would infer the 
existence of a new particle indirectly, the nature of which could be explored 
further with other indirect, as well as direct, methods.

The \ac{SM} cannot describe every phenomenon, yet there has been no conclusive 
deviation of experimental results from its predictions, and no searches for new 
particles have yielded results.
Further direct searches for new particles require a collider more powerful than 
those in the past, and indirect searches require larger data samples in order 
to be able to see small effects.
Both of these requirements can be met with the \acl{LHC}.
