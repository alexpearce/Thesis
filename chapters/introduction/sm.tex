\chapter{Testing the \acl{SM}}
\label{chap:intro:sm}

The \acl{SM} is a quantum field theory that describes the interactions of three 
of the four fundamental forces of nature: electromagnetism, the weak force, and 
the strong force.
It is able to accurately predict much of the observed phenomena in the 
universe, from the simple to the complex, despite its compact formulation.
Two notable phenomena the \ac{SM} does not describe are gravity, the fourth 
fundamental force of nature, and the imbalance between matter and anti-matter 
in the observable universe.
To address these shortcomings, theories \acl{BSM} have been proposed.

In addition to agreeing with existing experimental data, \ac{BSM} theories must 
also make new predictions that can be tested experimentally, either using 
existing data or with new experiments.
Often, \ac{BSM} theories predict the existence of new particles.
Their existence can be inferred experimentally in two ways: directly, where a 
new particle is produced and then decays in to two or more particles, which may 
themselves be new predictions of the theory; and indirectly, where the visible 
constituents of a process remain the same, but the new particles interact 
internally, changing some part of the interaction, such as the rate at which it 
occurs.

\cref{fig:intro:sm:particle_production:direct} illustrates one mechanism for 
the production of a muon and an anti-muon from the annihilation of an electron 
and a positron, denoted \decay{\Pelectron\APelectron}{\Pmuon\APmuon}.
The rate at which this process occurs can be precisely computed within the 
context of the \ac{SM}, and depends on the energies of the two electrons: when 
their total energy is equal to the mass of the \PZ boson, around 
\SI{91}{\GeV}~\cite{PDG2014}, the rate is high relative to the rate either side 
of this value.
This is a prediction of the theory: given two electrons interacting, such as in 
a particle collider, an experimenter will see a large increase in the number of 
muon pairs when the electrons have a combined energy around \SI{91}{\GeV}.
Seeing this behaviour in an experiment would validate the prediction of theory, 
and the \PZ boson would be then be said to have been `observed'.
This is an example of a direct observation of a particle.

\cref{fig:intro:sm:particle_production:indirect} shows one mechanism for the 
decay of a \PBs meson, a composite particle made of a bottom anti-quark and an 
strange quark, to a muon and an anti-muon, denoted \decay{\PBs}{\Pmuon\APmuon}.
The intermediary particles are all the other particles in the diagram, namely 
the up, charm, and top quarks, and the \PW bosons.
The \ac{SM} can also be used to predict the rate for this 
process~\cite{Bobeth:2013uxa}, which is equal to the probability that a given 
\PBs meson will decay to the $\Pmuon\APmuon$ final state, having a value around 
\num{3e-9}, or three \PBs mesons in every ten billion.
If a \ac{BSM} theory introduces new particles, they may contribute to the 
internal process in the diagram, either enhancing or reducing the rate.
An experimentalist can then count the number of \PBs mesons they see in total, 
and compare that with the number of \decay{\PBs}{\Pmuon\APmuon} decays they 
see.
If they see a statistically significant number of those decays larger than one 
per billion \PBs mesons, it is a sign that something beyond the \ac{SM} is 
contributing to the process.
If the experimentally observed rate is consistent with that predicted by a 
particular theory, there is evidence for the validity of that theory.

\begin{figure}
  \begin{subfigure}[b]{0.4\textwidth}
    \centering
    \feynmandiagram [large, horizontal=a to b] {%
  i1 [particle=\(\Pelectron\)] -- [fermion] a -- [fermion] i2 [particle=\(\APelectron\)],
  a -- [photon, edge label=\(\Pphoton/\PZ\)] b,
  f1 [particle=\(\APmuon\)] -- [fermion] b -- [fermion] f2 [particle=\(\Pmuon\)],
};

    \caption{Direct}
    \label{fig:intro:sm:particle_production:direct}
  \end{subfigure}
  \begin{subfigure}[b]{0.6\textwidth}
    \centering
    \begin{tikzpicture}
  \begin{feynman}
    \vertex (a1) {\APbottom};
    \vertex[right=1.5cm of a1] (a2);
    \vertex[right=1.5cm of a2] (a3);
    \vertex[right=1.5cm of a3] (a4) {\Pmuon};

    \vertex[below=4em of a1] (b1) {\Pdown};
    \vertex[right=1.5cm of b1] (b2);
    \vertex[right=1.5cm of b2] (b3);
    \vertex[right=1.5cm of b3] (b4) {\APmuon};

    \diagram* [large] {%
      {[edges=fermion]
        % Fermion line on LHS of box
        (b1) -- (b2) -- [edge label={\Pup,\Pcharm,\Ptop}] (a2) -- (a1),
        % Fermion line on RHS of box
        (b4) -- (b3) -- [edge label'={\Pup,\Pcharm,\Ptop}] (a3) -- (a4),
      },
      % Boson line at top of box
      (a2) -- [boson, edge label={\PW}] (a3),
      % Boson line at bottom of box
      (b2) -- [boson, edge label'={\PW}] (b3),
    };

    \draw [decoration={brace}, decorate] (b1.south west) -- (a1.north west)
    node [pos=0.5, left] {\PBs};
  \end{feynman}
\end{tikzpicture}

    \caption{Indirect}
    \label{fig:intro:sm:particle_production:indirect}
  \end{subfigure}
  \caption{%
    Feynman diagrams illustrating two processes through which the existence of 
    new particles may be inferred.
    \cref{fig:intro:sm:particle_production:direct} shows direct (resonant) 
    production of \PZ bosons, in the process 
    \decay{\Pelectron\APelectron}{\Pmuon\APmuon}.
    \cref{fig:intro:sm:particle_production:indirect} shows the process 
    \decay{\PBs}{\Pmuon\APmuon}, where a new particle can enter in the `box' in 
    the centre of the diagram.
  }
  \label{fig:intro:sm:particle_production}
\end{figure}

The \PZ boson was discovered in a similar process to that depicted in 
\cref{fig:intro:sm:particle_production:direct} in 
1983~\cite{1983398,BAGNAIA1983130}, and the decay \decay{\PBs}{\Pmuon\APmuon} 
was first observed in 2014.
The rate of both processes were consistent with \ac{SM} predictions.

Often, new particles are first seen using indirect methods, as was the case for 
the \PZ boson~\cite{HASERT1973138}, and their existence is then confirmed using 
direct methods.
The disadvantage of indirect methods is that it may not be conclusive what the 
nature of the new particle is, as it is interacting in somewhat subtle ways.
The advantage is that the energy of the incoming particles in the interaction 
can be much less than the mass of the new particle, due to the quantum 
mechanical nature of the internal interactions.
This can be seen in \cref{fig:intro:sm:particle_production:indirect}, where a 
\PW boson enters the diagram despite the mass of the \PBs meson being around 
fifteen times lighter~\cite{PDG2014}.

The \ac{SM} cannot describe every phenomenon, and yet everything that can be
predicted with it has stood against decades of experimental testing.
There is room for both direct and indirect searches for physics \acl{BSM}, and 
yet with no obvious bumps from new particles being seen at the \ac{LHC}, 
indirect searches seem like a particularly promising route to take.
