\chapter{The \acl{SM}}
\label{chap:intro:sm}

The \acf{SM} is a quantum field theory that describes the interactions of three 
of the four fundamental forces of nature: electromagnetism, the weak force, and 
the strong force.
These forces interact with and between a set of fundamental particles, 
particles which have no known substructure.
The behaviour of these particles can be observed or inferred experimentally to 
deduce the nature of the forces, such how their strength changes with distance.
This \namecref{chap:intro:sm} shall enumerate and describe these particles and 
forces, before discussing why additional theories are required and how they can 
be verified.

The \ac{SM} can be described equivalently by the interactions between fields or 
between particles, or by a mixture of the two.
In the particle formulation, forces are generated by the emission and 
absorption of force-carrier particles.
The \ac{EM} force acts between particles with a non-zero electric charge via 
exchange of photons, particles of light.
The electric charge is an example of the \emph{conserved charge} that is 
associated with a field, the \ac{EM} field in this case.
Interactions with a field preserve the total associated charge of the system, 
and the field does not interact with particles with a charge of zero.

The strong force acts on particles with a non-zero \emph{colour charge} via the 
exchange of gluons, with quarks and gluons being the only particles that have a 
non-zero colour charge.
Gluons being charged under the force they mediate significantly complicates the 
theory of strong interactions, called \ac{QCD}, in comparison with \ac{EM} 
interactions.
At low energies, this self interaction leads to the phenomenon of confinement, 
whereby colour-charged objects cannot be observed experimentally, as the 
attractive force between two colour-charged object is constant with increasing 
distance, requiring an infinite amount of energy to separate them completely.
The six quarks, given the names up, down, charm, strange, top, and bottom 
(\Pup, \Pdown, \Pcharm, \Pstrange, \Ptop, and \Pbottom, generically \Pquark), 
are only observed in colourless bound states of two or more quarks called 
hadrons.
Mesons are hadrons made of two quarks, and baryons are hadrons made of three 
quarks.
The family of baryons includes the protons and the neutron, made up of 
$\Pup\Pup\Pdown$ and $\Pup\Pdown\Pdown$ quarks respectively, which together 
make up the bulk of the mass of atoms.
The binding of protons and neutrons together in an atomic nucleus is mediated 
indirectly by the strong force.

The last of the three forces described by the \ac{SM} is the weak force, which 
interacts with particles with a non-zero \emph{weak isospin} \wisospin.
The carriers of the weak force are the two charged \PWpm bosons and the neutral 
\PZ boson.
It is responsible for the nuclear decay of radioactive atoms.
The quarks can be categorised by their weak isospin values in to the up-type 
quarks with $\wisospin = \sfrac{1}{2}$, up, charm, and top, and the down-type 
quarks with $\wisospin = -\sfrac{1}{2}$, down, strange, and bottom.
Likewise, the leptons, another family of fundamental particles, can also be 
categorised by their weak isospin values into the electron-like particles with 
$\wisospin = -\sfrac{1}{2}$, electron, muon, and tau particle, and the 
neutrino-like particles with $\wisospin = \sfrac{1}{2}$, electron neutrino, 
muon neutrino, and tau neutrino (\Pelectron, \Pmuon, \Ptau, \Pnue, \Pnum, and 
\Pnut).
The quarks and leptons are often written as \emph{doublets} of particles, pairs 
with opposite values of the weak isospin, such as
\begin{equation}
  \begin{pmatrix}\Pup\\\Pdown\end{pmatrix}
  \text{\ and\ }
  \begin{pmatrix}\Pnue\\\Pelectron\end{pmatrix}.
  \label{intro:sm:weak_doublets}
\end{equation}
Particles within these doublets can interact via the exchange of \PWpm bosons, 
and it is this mechanism that permits the radioactive decay of atoms, wherein a 
neutron decays into a proton, an electron, and an electron neutrino by the 
decay of one of the down quarks in the neutron..
This process is illustrated in \cref{fig:intro:sm:neutron_decay} as a 
\emph{Feynman diagram}, which represents particles as lines and interactions 
between particles as vertices.

\begin{figure}
  \centering
  % \begin{tikzpicture}
%   \begin{feynman}
%     \vertex (a);
%     \vertex [left=0.1cm of a] (a1);
%     \vertex [left=0.1cm of a1] (a11);
%     \vertex [right of=a] (b);
%     \vertex [above of=b] (c);
%     \vertex [below left of=a] (i1);
%     \vertex [left=0.1cm of i1] (i11);
%     \vertex [left=0.1cm of i11] (i111);
%     \vertex [above left of=a] (i2);
%     \vertex [left=0.1cm of i2] (i21);
%     \vertex [left=0.1cm of i21] (i211);
%     \vertex [left=0.3cm of c] (f1) {\(\bar{\nu}_{e}\)};
%     \vertex [right=0.3cm of c] (f2) {\(e^{-}\)};
%     \diagram*[small] {%
%       (i1) -- [fermion, edge label'=\(d\)] (a) -- [fermion, edge label'=\(u\)] (i2),
%       (i11) -- (a1) -- (i21),
%       (i111) -- (a11) -- (i211),
%       (a) -- [boson, edge label=\(W^{-}\)] (b),
%       (f1) -- [fermion] (b) -- [fermion] (f2),
%     };
%   \end{feynman}
% \end{tikzpicture}
\feynmandiagram [layered layout, horizontal=a to b] {%
  a [particle=\(\Pdown\)] -- [fermion] b -- [fermion] f1 [particle=\(\Pup\)],
  b -- [boson, edge label'=\(\PWm\)] c,
  c -- [anti fermion] f2 [particle=\(\APnue\)],
  c -- [fermion] f3 [particle=\(\Pelectron\)],
};

  \caption{%
    Feynman diagram for the decay of a down quark in to an up quark, an 
    electron, and an electron neutrino, \decay{\Pdown}{\Pup\Pelectron\APnue}.
    The process conserves charge, lepton number, and weak isospin.
  }
  \label{fig:intro:sm:neutron_decay}
\end{figure}

Each of the fundamental particles described so far can each have an associated 
antiparticle, which is nearly identical to the particle but has the opposite 
electric charge.
The neutral photon, gluon, and \PZ boson are each their own antiparticles, and 
the \PWp and \PWm bosons are each other's antiparticles.
For the other particles, the antiparticle is denoted with a bar, such as the 
anti-up quark \APup, or with the opposite charge, such as the anti-electron 
\APelectron (also called the positron).
The \ac{SM} predicts that particles and antiparticles behave very similarly, to 
the extent that it is an open question as to why the visible universe is 
dominated by matter, given that the \ac{SM} predicts similar rates of matter 
and antimatter production in the early universe.
This problem will be discussed in more detail in \cref{chap:cpv:theory}.

\section{Searching for new particles}

The \acl{SM} is able to accurately predict much of the observed phenomena in 
the universe, from the simple to the complex, despite its compact formulation.
Two notable phenomena the \ac{SM} does not describe are gravity, the fourth 
fundamental force of nature, and the aforementioned imbalance between matter 
and antimatter in the observable universe.
These problems, and others, prevent a complete understanding of the physical 
world, and so they must be addressed.
To this end, theories \acl{BSM} have been proposed.

In addition to agreeing with existing experimental data, \ac{BSM} theories must 
make new predictions that can be tested experimentally, either using existing 
data or with new experiments.
Often, \ac{BSM} theories predict the existence of new particles.
Their existence can be inferred experimentally in two ways: directly, where a 
new particle is produced and then decays into two or more particles, which may 
themselves be new predictions of the theory; and indirectly, where the visible 
constituents of a process remain the same, but the new particles interact 
internally, changing some part of the interaction, such as the rate at which it 
occurs.

\cref{fig:intro:sm:particle_production:direct} illustrates one mechanism for 
the production of a muon and an anti-muon from the annihilation of an electron 
and a positron, denoted \decay{\Pelectron\APelectron}{\Pmuon\APmuon}.
The rate at which this process occurs can be precisely computed within the 
context of the \ac{SM}, and depends on the energies of the two electrons: when 
their total energy is equal to the mass of the \PZ boson, around 
\SI{91}{\GeV}~\cite{PDG2014}, the rate is high relative to the rate either side 
of this value.
This is a prediction of the theory: given two electrons interacting, such as in 
a particle collider, an experiment will measure a large increase in the number 
of muon pairs when the electrons have a combined energy around \SI{91}{\GeV}.
Seeing this behaviour in an experiment would validate the prediction of theory, 
and the \PZ boson would be then be said to have been `observed'.
This is an example of a direct observation of a particle.

\cref{fig:intro:sm:particle_production:indirect} shows one mechanism for the 
decay of a \PBs meson, a composite particle made of a bottom anti-quark and a 
strange quark, to a muon and an anti-muon, denoted \decay{\PBs}{\Pmuon\APmuon}.
The intermediary particles are all the other particles in the diagram, namely 
the up, charm, and top quarks, and the \PW bosons.
The \ac{SM} can also be used to predict the rate for this 
process~\cite{Bobeth:2013uxa}, which is equal to the probability that a given 
\PBs meson will decay to the $\Pmuon\APmuon$ final state, having a value around 
\num{3e-9}, or three \PBs mesons in every ten billion.
If a \ac{BSM} theory introduces new particles, they may contribute to the 
internal process in the diagram, either enhancing or reducing the rate.
An experimentalist can then count the total number of \PBs mesons observed, and 
compare that with the number of \decay{\PBs}{\Pmuon\APmuon} decays.
If they see a significantly larger number of those decays than one per billion 
\PBs mesons, it is a sign that something beyond the \ac{SM} is contributing to 
the process.
If the experimentally observed rate is consistent with that predicted by a 
particular theory, there is evidence for the validity of that theory.

\begin{figure}
  \begin{subfigure}[b]{0.4\textwidth}
    \centering
    \feynmandiagram [large, horizontal=a to b] {%
  i1 [particle=\(\Pelectron\)] -- [fermion] a -- [fermion] i2 [particle=\(\APelectron\)],
  a -- [photon, edge label=\(\Pphoton/\PZ\)] b,
  f1 [particle=\(\APmuon\)] -- [fermion] b -- [fermion] f2 [particle=\(\Pmuon\)],
};

    \caption{Direct}
    \label{fig:intro:sm:particle_production:direct}
  \end{subfigure}
  \begin{subfigure}[b]{0.6\textwidth}
    \centering
    \begin{tikzpicture}
  \begin{feynman}
    \vertex (a1) {\APbottom};
    \vertex[right=1.5cm of a1] (a2);
    \vertex[right=1.5cm of a2] (a3);
    \vertex[right=1.5cm of a3] (a4) {\Pmuon};

    \vertex[below=4em of a1] (b1) {\Pdown};
    \vertex[right=1.5cm of b1] (b2);
    \vertex[right=1.5cm of b2] (b3);
    \vertex[right=1.5cm of b3] (b4) {\APmuon};

    \diagram* [large] {%
      {[edges=fermion]
        % Fermion line on LHS of box
        (b1) -- (b2) -- [edge label={\Pup,\Pcharm,\Ptop}] (a2) -- (a1),
        % Fermion line on RHS of box
        (b4) -- (b3) -- [edge label'={\Pup,\Pcharm,\Ptop}] (a3) -- (a4),
      },
      % Boson line at top of box
      (a2) -- [boson, edge label={\PW}] (a3),
      % Boson line at bottom of box
      (b2) -- [boson, edge label'={\PW}] (b3),
    };

    \draw [decoration={brace}, decorate] (b1.south west) -- (a1.north west)
    node [pos=0.5, left] {\PBs};
  \end{feynman}
\end{tikzpicture}

    \caption{Indirect}
    \label{fig:intro:sm:particle_production:indirect}
  \end{subfigure}
  \caption{%
    Feynman diagrams illustrating two processes through which the existence of 
    new particles may be inferred.
    \Cref*{fig:intro:sm:particle_production:direct} shows direct (resonant) 
    production of \PZ bosons, in the process 
    \decay{\Pelectron\APelectron}{\Pmuon\APmuon}.
    \Cref*{fig:intro:sm:particle_production:indirect} shows the process 
    \decay{\PBs}{\Pmuon\APmuon}, where a new particle can enter in the `box' in 
    the centre of the diagram.
  }
  \label{fig:intro:sm:particle_production}
\end{figure}

The \PZ boson was discovered in a similar process to that depicted in 
\cref{fig:intro:sm:particle_production:direct} in 
1983~\cite{1983398,BAGNAIA1983130}, and the decay \decay{\PBs}{\Pmuon\APmuon} 
was first observed in 2014.
The rate of both processes were consistent with \ac{SM} predictions.

Often, new particles are first seen using indirect methods, as was the case for 
the \PZ boson~\cite{HASERT1973138}, and their existence is then confirmed using 
direct methods.
The disadvantage of indirect methods is that it may not be conclusive what the 
nature of the new particle is, as it is interacting in somewhat subtle ways.
The advantage is that the energy of the incoming particles in the interaction 
can be much less than the mass of the new particle, due to the quantum 
mechanical nature of the internal interactions.
This can be seen in \cref{fig:intro:sm:particle_production:indirect}, where a 
\PW boson enters the diagram despite the mass of the \PBs meson being around 
fifteen times lighter~\cite{PDG2014}.

The \ac{SM} cannot describe every phenomenon, and yet everything that can be
predicted with it has stood against decades of experimental testing.
There is room for both direct and indirect searches for physics \acl{BSM}, and 
yet with no obvious bumps from new particles being seen at the \ac{LHC}, 
indirect searches seem a particularly promising route to take.
