\chapter{Fitting}
\label{chap:prod:fitting}

Charm mesons produced directly from the proton-proton interaction are referred 
to here as \emph{prompt}, whereas those produced in the decays of \PB hadrons 
are referred to as \emph{secondary}.
Secondary charm is treated as a background, and its contribution to the dataset 
used for the measurement must be accounted for.
Prompt charm hadrons can be produced either directly in the state that is 
reconstructed, such as \PDzero, or from feed-down of excited charm states.
No attempt is made to distinguish between these two categories of prompt charm.
For each mode, the sum of the prompt and secondary charm meson yield in each 
\pTy\ bin is measured using an extended maximum likelihood fit to the charm 
hadron mass spectrum.
These fits are performed simultaneously in all \pTy\ bins to allow model 
parameters to be shared across the bins, where appropriate.
The contribution of prompt and secondary charm is then disentangled in a 
separate fit to the distribution of the natural logarithm of the impact 
% TODO: maybe we'll have an 'analysis commonalities' section where this will
% live
parameter \chisq, \lnipchisq, as described in \cref{chap:intro:lhcb}.
