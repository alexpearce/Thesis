\chapter{Event and candidate selection}
\label{chap:prod:sel}

The decay modes chosen for this analysis,
\DzToKpi, \DpToKpipi, \DspTophipi, \DstToDzpi with \DzToKpi, and their charge 
conjugates, are the most probable fully charged, hadronic final states for each 
meson.
The advantage of choosing such final states is that it allows the invariant 
mass of the charm hadrons to be computed unambiguously, so that a yield 
determination can be performed via fits to the mass distribution.
These fits shall be discussed in detail in \cref{chap:prod:fitting}, but it is 
worth noting at this point that such fits are simpler if the data is `clean', 
that is if the fraction of charm meson signal in the data is high.
It is then advantageous to optimise the selection of the data to reduce 
backgrounds, while retaining as much signal as possible to maximise the 
statistical significance.

The selection of fully charged, hadronic decay modes of charm mesons has 
already been performed by many other analysis at \lhcb,\footnotemark and their 
selections share several common features.
The first of these is a requirement on the minimum charm candidate \pT, as 
there are many soft particles produced directly in the proton-proton collision, 
and such a requirement removes many of them.
The second is to ask that the charm candidate vertex is significantly displaced 
from the proton-proton collision vertex, or \ac{PV}, exploiting the long flight 
distances of charm hadrons in the laboratory frame.
Finally, it is required that the \ipchisq\ of the charm meson, a measure of the 
\acf{IP} relative to the uncertainty on the \ac{IP}, does not exceed a certain 
value.
This suppresses the background of charm candidates whose three-momentum does 
not point back to the \ac{PV}, which the momentum of random combinations of 
tracks will not, on average.
Each of these three features are powerful discriminators between signal and 
background.

For this measurement of prompt charm production, it is preferable not to make 
\pT\ requirements on the charm candidate, as they would restrict the \pT range 
available for differential measurements.
Nor is it desirable to make \ipchisq\ cuts, as the \ipchisq\ distribution is 
used for prompt-secondary discrimination, and having access to only part of the 
distribution can hinder accurate modelling.
The selection used is then informed by previous \lhcb\ studies of fully 
charged, hadronic charm decays, but with modifications to improve the 
usefulness for the measurement in hand.
The selection variables that are useful for discriminating between signal and 
background, in addition to those previously mention, are given below.
The `parent' is the charm hadron \PHc, and the `children' are the charged 
hadrons in the final state.

\begin{description}
  \item[Child track \chisq\ per degree of freedom] \hfill \\
    The \chisq\ per degree of freedom of the child track returned by the track 
    fit.
    This variable discriminates between real and fake~(ghost) tracks.
  \item[Child transverse momentum] \hfill \\
    Momentum transverse to the $z$-axis.
    Children of charm hadrons will have a higher \pT\ on average than prompt 
    stable particles produced directly from the \pp\ collision.
  \item[Child momentum] \hfill \\
    Three-momentum magnitude \ptot.
    Must be within the range of acceptable particle identification performance 
    and fall within the kinematic spectrum of the \ac{PID} calibration samples, 
    discussed in \cref{chap:prod:effs:pid}.
  \item[Child pseudorapidity] \hfill \\
    Pseudorapidity \Eta, defined in \cref{eqn:intro:lhcb:pseudorapidity}.
    Must be within the range of acceptable particle identification performance 
    and fall within the kinematic spectrum of the \ac{PID} calibration samples.
  \item[Child \ipchisq] \hfill \\
    Impact parameter with respect to the \ac{PV}, as shown 
    in~\cref{fig:intro:lhcb:vertexing}.
    Prompt stable particles produced directly in the \pp\ collision will have a 
    low impact parameter, whereas particles produced in long-lived charm decays 
    will not have trajectories pointing back to the \ac{PV} on average.\\
    The `\chisq' is computed as $\Delta^{\text{T}}\Sigma^{-1}\Delta$, where 
    $\Delta$ is the difference in the position vectors of the track and the 
    vertex, and $\Sigma$ is the sum of the covariance matrices of the track and 
    vertex fits.
    This quantity is approximately equal to the ratio of the \ac{IP} and the 
    uncertainty on that \ac{IP} measurement.
  \item[Child \ac{PID}] \hfill \\
    Particle identification information as the combined response of the 
    calorimeters, muon stations, and \rich\ detectors.
    % TODO this will probably be explained in the detector chapter
    Predicted Cherenkov rings from tracks reconstructed by the tracking system 
    are compared with the photons detected by the {\rich}s.
    A global likelihood fit is then performed across all radiators and tracks, 
    and a likelihood, relative to the pion hypothesis, is assigned to each 
    track for each particle hypothesis.
  \item[Pairwise child \ac{DOCA}] \hfill \\
    For a set of tracks physically originating from a common vertex, the 
    largest distance between any two tracks, the \ac{DOCA}, should be small.
  \item[Pairwise child \ac{DOCA} \chisq] \hfill \\
    This variable is used in one trigger line and relates to \ac{DOCA} as 
    \ipchisq\ relates to \ac{IP}.
  \item[Parent reconstructed mass] \hfill \\
    Invariant mass, either computed before the vertex fit as the sum of the 
    child four-momenta, or after as the mass of the fitted vertex.
    In a signal sample, a peaking structure will be seen around the nominal 
    charm hadron invariant mass.
    To confidently describe the combinatorial background, the window around the 
    nominal mass should be many times wider than the resolution on this peak.
  \item[Parent vertex quality] \hfill \\
    Quality of the vertex fit.
    A larger \chisq\ value means a lower probability that the input tracks 
    originated from the same spatial point, which would be true if the tracks 
    were the product of a particle decay.
  \item[Parent vertex displacement] \hfill \\
    Displacement of the decay vertex from the production vertex, assumed to be 
    the \ac{PV}, as either the reconstructed lifetime $c\lifetime$ or the 
    vertex displacement \chisq.
    The `\chisq' is computed as $\Delta^{\text{T}}\Sigma^{-1}\Delta$, where 
    $\Delta$ is the difference in the spatial positions of the two vertices, and 
    $\Sigma$ is the sum of the covariance matrices of the two vertex fits.
  \item[Parent \ac{DIRA}] \hfill \\
    If the child tracks are from a true decay, and the decay vertex position is 
    correctly reconstructed, then the momentum vector of the parent should 
    point in the same direction as the displacement vector between the 
    production vertex and the decay vertex.
    The angle between the momentum and displacement vectors, the \ac{DIRA}, 
    should be small for a correctly reconstructed signal decay.
\end{description}

\footnotetext{%
  Representative examples include selections of 
  \decay{\PDzero}{\PKmp\Ppipm}~\cite{Aaij:2012nva}, 
  \decay{\PDp}{\PKminus\PKplus\Ppiplus}~\cite{Aaij:2011cw}, and 
  \DspTophipi~\cite{Aaij:2012cy} decays.
}
