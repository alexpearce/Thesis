\chapter{Summary}
\label{chap:prod:summary}

Measurements of prompt open charm production cross-sections can be used to 
constrain \aclp{QCDPDF}, a key component in \ac{QCD} predictions of 
cross-sections at the \ac{LHC} and beyond.
They can also be used to set limits on the background rates in atmospheric 
neutrino experiments, and are used by the experiments at the \ac{LHC} to 
estimate the rates of very rare processes and to plan for future data-taking 
conditions.

This \namecref{chap:prod} presented the measurements of prompt production 
cross-sections of the open charm mesons \PDzero, \PDplus, \PDsplus, and 
\PDstarp\ with the \lhcb\ detector, using data taken at a proton-proton 
centre-of-mass energy of \sqrtseq{13}.
The measurements are made in bins of charm hadron transverse momentum and 
rapidity.
A significantly higher precision was achieved in comparison to complementary 
\lhcb\ measurements made at \sqrtseq{7}, due to a increase in integrated 
luminosity of around 300 times with $\intlumi = 
\SI{\xsectotlumi}{\per\pico\barn}$.
The data were able to be analysed immediately after the data-taking thanks to 
the Turbo stream processing model, unique to the \lhcb\ experiment and new for 
\runtwo\ of the \ac{LHC}.
A precise luminosity measurement was able to be made due to the employment of 
the \acl{BGI} method, only possible due to the high precision of the \lhcb\ 
vertex detector.

In comparison with a set of theoretical predictions, using different inputs and 
having varying kinematic ranges and uncertainties, the data are found to be in 
agreement although generally lie at the upper edge of the uncertainty 
intervals.
This indicates that the \pTy\ dependence of the cross-sections are 
well-modelled, but that the absolute normalisation is too low.
Similar levels of agreement are seen between predictions and measurements of 
\SI{13}{\TeV} cross-sections relative to those measured at \sqrtseq{7}, and for 
integrated \ccbar\ cross-sections.

It is found that the assumption of universality of charm fragmentation 
fractions $f(\cToHc)$ is valid, as measurements of relative cross-sections 
between charm mesons agree with similar measurements made at the \bfactories.

The study of charm physics has the potential to uncover the first hints of 
physics beyond the \ac{SM}, although such potential relies on the collection of 
vast datasets in order to resolve what must be very small effects.
With this measurement, it is known that around \num{3e12} \ccbar\ pairs will be 
produced within acceptance of the \lhcb\ detector for every 
\SI{1}{\per\femto\barn} of integrated luminosity detected, and that the \lhcb\ 
detector is capable of cleanly and efficiently reconstructing open charm 
mesons.
Despite its name, \lhcb\ is a charm factory.
