\chapter{Systematic uncertainties}
\label{chap:prod:syst}

Each experimental measurement of the double-differential cross-section defined 
in \cref{eqn:prod:introduction:differential_cross_section} is comprised of a 
central value and a set of uncertainties.
One uncertainty is \emph{statistical}, and is taken to be the uncertainty on 
the prompt signal yield $N(\HcTof)$, as measured in the fitting procedure 
described in \cref{chap:prod:fitting}.
The second uncertainty is the \emph{systematic} uncertainty, which quantifies 
all other sources of experimental precision.
This section will describe the individual components of this uncertainty, how 
they arise and how their values are estimated, and will finish with a summary 
stating how the components are combined as a single number.

% TODO talk about how these uncertainties relate to the MC propagation we've 
% already talked about in the intro
The statistical treatment of systematic uncertainties is similar to that

\section{Finite \acl{MC} sample size}
\label{chap:prod::syst:mcstat}

The Monte Carlo samples used to estimate the acceptance, reconstruction, and 
selection efficiencies are of a finite size, and so the computed efficiencies 
carry a non-zero statistical uncertainty.
This is propagated to the cross-section measurements as a systematic 
uncertainty.
% TODO what about D0 effs using the D*+ MC?
As the set of \ac{MC} events is unique for each \pTy\ bin and mode, this 
systematic uncertainty is fully uncorrelated between bins and between modes.
As discussed in \cref{chap:prod:effs:tot}, the posteriori probability 
for the estimated efficiency is described by the beta distribution, and so 
the corresponding nuisance parameter is modelled by this distribution after 
shifting it such that the most probable value coincides with zero.

The resulting relative systematic uncertainties are given in 
Tables~\ref{tab:syst:mc:stat:DzToKpi},
\ref{tab:syst:mc:stat:DpToKpipi},
\ref{tab:syst:mc:stat:DsTophipi}, and
\ref{tab:syst:mc:stat:DstToDzpi_DzToKpi} for \DzToKpi, \DpToKpipi, \DspTophipi, 
and \DstToDzpi\ with \DzToKpi.

\begin{table}
  \caption{%
    Relative uncertainty on the \PDzero cross-section, in \PDzero \pTy\ bins, 
    due to the finite size of the Monte Carlo sample used to assess the 
    selection efficiency.
  }
  \label{tab:prod:syst:mc:stat:D0ToKpi}
  \centering
  % \input{tables/systematics/D0ToKpi/mc_stat}
\end{table}

\section{\acl{MC} modelling}
\label{chap:prod:syst:mc}

% TODO create MC-data comparison plots
% Need to run create_weight_tree in tracking_mcerp on the modes to create 
% tracking weights

If the simulated data does not correctly model the selection variables, the 
selection efficiencies that are evaluated with \ac{MC}, described in 
\cref{chap:prod:effs:sel}, will be incorrect.
To assess the accuracy of the \acl{MC} modelling, signal distributions are 
extracted from the signal using the \sPlot\ method~\cite{Pivk:2004ty} and are 
compared to the \ac{MC}.
As the selection efficiency does not include the effects \ac{PID} selection, the data distributions are corrected for the \ac{PID} efficiency by a per-event weighting.
The signal distributions in data are only available after the selection has been applied, and so the same selection is also applied to the simulated data (with the exception of the \ac{PID} requirements which are treated separately).
The resulting histograms are shown in the Figures in 
Appendix~\label{app:mcdata}.

To assess the effect of any mis-modelling on the selection efficiencies, the 
following procedure is performed for each selection variable $x$:
\begin{enumerate}
  \item Find the cut value $y$ on $x$ which rejects half of the simulated data;
  \item Apply the cut value $y$ on $x$ to the \emph{real data}, counting the 
    number of signal candidates passing the requirement as the sum of signal 
    weights divided by the \ac{PID} selection efficiencies
    \begin{equation}
      N_{\text{Passed}} = \sum_{N}^{i} I(i)\frac{w_{i,\text{signal}}}{\effpid},
    \end{equation}
    where $I(i)$ is 1 if the value of $x$ for the $i$th event passes the 
    requirement on $x$, and is zero otherwise;
  \item The efficiency-corrected signal yield corresponding to the requirement 
    on $x$ is $2N_{\text{Passed}}$, which should equal the number of signal 
    candidates before the requirement $y$, if the \ac{MC} describes the data.  
    The quantity
    \begin{equation}
      \Delta = 2N_{\text{Passed}} - \sum_{N}^{i} \frac{w_{i,\text{signal}}}{\effpid},
      \label{eqn:prod:syst:mc:eff_diff}
    \end{equation}
    is then a handle on the mis-modelling of the data in the \ac{MC}.
\end{enumerate}
The signal weights are the product of signal sWeights, computed using the \sPlot\ method, and weights that equalise the \pT\ and \rapidity\ distributions between the data and the \ac{MC}.

To account for correlations between variables, which may lead to 
double-counting of the mis-modelling if neglected, the procedure is repeated 
in ten thousand pseudo-experiments per selection variable.
For each trial, each candidate is sampled $n$ times from the real data, with $n$ being randomly sampled from a Poisson distribution of mean 1.
The total systematic uncertainty on the efficiency-corrected signal yield due 
to \ac{MC} mis-modelling is the mean of the set of ten thousand values of $\Delta$, as defined in \cref{eqn:prod:syst:mc:eff_diff}.

As the \ac{MC} sample sizes are too small to make precise differential 
measurements of this uncertainty, the above procedure is performed on the data 
integrated across all \pTy\ bins.
The \pTy-integrated systematic uncertainty is given for each mode in 
Table~\ref{tab:prod:syst:mc:result}

\begin{table}
  \caption{%
    Relative uncertainty on the \PDzero cross-section, in \PDzero \pTy\ bins, 
    due to the finite size of the Monte Carlo sample used to assess the truth 
    matching efficiency.
  }
  \label{tab:prod:syst:mc:result}
  \centering
  \begin{tabular}{lr}
  \toprule
  Decay mode                & Relative error     \\
  \midrule
  \DzToKpi                  & \SI{1.0}{\percent} \\
  \DpToKpipi                & \SI{0.3}{\percent} \\
  \DspTophipi               & \SI{0.4}{\percent} \\
  \DstToDzpi\ with \DzToKpi & \SI{2.5}{\percent} \\
  \bottomrule
\end{tabular}

\end{table}

\section{Monte Carlo truth matching}
\label{chap:prod:syst:mc:truth_matching}

The truth matching efficiency described in \cref{chap:prod:effs:truth} is computed using an \ac{MC} 
sample of finite size, and so carries a statistical uncertainty.
As in \cref{chap:prod::syst:mcstat}, this statistical uncertainty is propagated to the cross-section 
measurements as a systematic uncertainty.
Unlike the selection efficiency uncertainty, the \ac{MC} sample size is too 
small to compute the truth matching systematic in \pTy\ bins, and so an 
integrated value is computed.
The values for each mode are given in Table~\ref{tab:prod:syst:mc:truth_matching}.

% TODO this table isn't in the 13 TeV ANA
% can make it using the run_studies script, look in the split_uncertainties 
% folder
\begin{table}
  \caption{%
    Relative uncertainty on the \PDzero cross-section, in \PDzero \pTy\ bins, 
    due to the finite size of the Monte Carlo sample used to compute the truth 
    matching efficiency.
  }
  \label{tab:prod:syst:mc:truth_matching}
  \centering
  \begin{tabular}{lr}
  \toprule
  Decay mode                & Relative error      \\
  \midrule
  \DzToKpi                  & \SI{0.02}{\percent} \\
  \DpToKpipi                & \SI{0.18}{\percent} \\
  \DspTophipi               & \SI{0.11}{\percent} \\
  \DstToDzpi\ with \DzToKpi & \SI{0.08}{\percent} \\
  \bottomrule
\end{tabular}

\end{table}


\section{Particle identification calibration}
\label{chap:prod:syst:pid}

Four common sources are typically mentioned when discussing systematic 
uncertainties in the PIDCalib approach:
\begin{enumerate}
  \item Mis-modelling in the reference sample: the chosen reference
    sample does not correctly model the kinematics. This does not apply
    to this analysis as a purely data-driven approach is used where the
    actual data is reweighted.
  \item Uncertainty due to the \sPlot\ technique in the calibration sample:
    We perform fits in each calibration bin separately, hence this does not 
    apply here.
  \item Limited sample size of the calibration sample: This is incorporated
    as described before by bootstrapping the calibration sample before each
    fit and including the variation of the obtained efficiency in each as 
    part of the Monte Carlo error propagation.
  \item The chosen binning scheme can bias the efficiency if the distributions 
    of the reference and calibration sample within a bin are significantly
    different. The treatment of this uncertainty will be discussed in the following.
\end{enumerate}

To estimate the effect of the binning systematic, we employ an alternative method using
the boosted decision tree based reweighting algorithm of the \texttt{hep\_ml} package.
An efficiency can be seen as the weight that as the weight that has to be assigned
to an event before the selection criteria to have the obtained distribution agree
with the one obtained after the cut.

Therefore, we trained the BDT classifier to discriminate between the 
distributions in the calibration sample before and after the \dllkpi\ cut being 
applied to the track. As
training variables, we use \ptot, \Eta, and \spd, while additional variables 
can be
included, no significant change was observed if $\phi$, \pT, or the number of 
tracks were
added in additional to the nominal three variables.

The trained reweighter is then used to predict the efficiencies for every event in the
reference sample and the efficiency in each \pTy\ bin is obtained as before.  
The difference
in obtained efficiency to the nominal method using a binned calibration sample is then taken
as the systematic uncertainty.

The resulting relative systematic uncertainties due to the finite size of the 
\ac{PID} calibration sample are given in 
Tables~\ref{tab:syst:pid:stat:DzToKpi},
\ref{tab:syst:pid:stat:DpToKpipi},
\ref{tab:syst:pid:stat:DsTophipi}, and
\ref{tab:syst:pid:stat:DstToDzpi_DzToKpi} for \DzToKpi, \DpToKpipi, 
\DspTophipi, and \DstToDzpi with \DzToKpi.
The corresponding tables for the relative systematic uncertainties due to the 
choice of binning are given in Tables~\ref{tab:syst:pid:binning:DzToKpi},
\ref{tab:syst:pid:binning:DpToKpipi},
\ref{tab:syst:pid:binning:DsTophipi}, and
\ref{tab:syst:pid:binning:DstToDzpi_DzToKpi}.

\begin{table}
  \caption{%
    Relative uncertainty on the \PDzero cross-section, in \PDzero \pTy\ bins, 
    due to the finite size of the \ac{PID} calibration sample.
  }
  \label{tab:syst:pid:stat:DzToKpi}
  \centering
  % \input{tables/systematics/D0ToKpi/pid_stat}
\end{table}

\begin{table}
  \caption{%
    Relative uncertainty on the \PDzero cross-section, in \PDzero \pTy\ bins, 
    due to the choice of \ac{PID} calibration binning.
  }
  \label{tab:syst:pid:binning:DzToKpi}
  \centering
  % \input{tables/systematics/D0ToKpi/pid_bin}
\end{table}


\section{Luminosity}
\label{chap:prod:syst:lumi}

There is an uncertainty on the luminosity measurement provided by the 
luminosity group, which we will propagate as a systematic uncertainty on our 
measurement.
The relative uncertainty on the luminosity measurement is expected to be around 
$5\%$.

\section{Tracking efficiency correction}
\label{chap:prod:syst:tracking}

There is an uncertainty on the tracking correction, discussed in 
\cref{chap:prod:effs:tracking}, provided by the tracking group, which we will 
propagate as a systematic uncertainty on our measurement.
There are several potential sources for systematic uncertainties:
\begin{itemize}
  \item The uncertainties given in \cref{fig:prod:effs:tracking_table} which incorporate the 
    statistical uncertainties from the calibration mode and differences between different methods
    used to extract these factors.
  \item The correction table provides numbers estimated using muons. An additional uncertainty
    of \SI{1.1}{\percent} (\SI{1.4}{\percent}) is applied for kaons (pions) due 
    to uncertainties in the hadronic interactions.
    This is dominated by the uncertainty on the material budget and hence full correlation
    between kaons and pions is assumed.
  \item An additional uncertainty of \SI{0.4}{\percent} is applied due to the 
    reweighting procedure.
\end{itemize}
All of these sources are propagated using Monte Carlo error propagation and assuming 
Gaussian distributions with the above mentioned relative widths for the different
contributions to the tracking systematic uncertainty.
% a toy study. Per toy, a new value for each of these
% contributions is randomly drawn from a Gaussian distribution, taking into account the known
% correlation for the hadronic interaction uncertainty. The calibration procedure is 
% repeated for each of these toys. The overall uncertainty is then given by the standard deviation
% of the values obtained from 10000 toy experiments. A common set of toy calibration histograms is generated
% which is used for determining the uncertainty in each bin and for each mode, allowing to determine the correct correlations.
Even though different bins and modes have different final state kinematics and hence access the table of correction
factors for each track differently, the obtained uncertainty is driven by the global factors for the reweighting and the hadronic
interaction uncertainty and the results are found to be highly correlated (at 
least \SI{90}{\percent} between different bins and modes.


The resulting relative systematic uncertainties are given in 
Table~\ref{tab:syst:tracking:DzToKpi} for \DzToKpi, in 
Table~\ref{tab:syst:tracking:DpToKpipi} for \DpToKpipi, in 
Table~\ref{tab:syst:tracking:DsTophipi} for \DspTophipi, and in 
Table~\ref{tab:syst:tracking:DstToDzpi_DzToKpi} for \DstToDzpi with \DzToKpi.

The posteriori distribution obtained from the toy propagation of the uncertainties is then used as the distribution
for the associated nuisance parameter after shifting it such that the most probable value coincides with zero.

\begin{table}
  \caption{%
    Relative uncertainty on the \PDzero cross-section, in \PDzero \pTy\ bins, 
    due to the finite sample size and hadronic interaction uncertainty on the 
    tracking efficiency correction measurement.
  }
  \label{tab:syst:tracking:DzToKpi}
  \centering
  % \begin{tikzpicture}[
  x={(0.866cm, -0.5cm)}, y={(0.866cm, 0.5cm)}, z={(0cm, 1cm)},
  scale=0.9,
  inner sep=0pt, outer sep=2pt,
  axis/.style={thick,->},
  station/.style={fill=black!40!white, opacity=0.3},
  particle/.style={->, dashed},
  hit/.style={star,star points=7,draw=black!70,fill=red!70,inner sep=0pt,minimum size=0.2cm}
  ]

  % Coordinate axes
  \coordinate (O) at (0, 0, 0);
  \draw[axis] (O) -- +(14, 0, 0) node [right] {$z$};
  \draw[axis] (O) -- +(0, 2.5, 0) node [right] {$x$};
  \draw[axis] (O) -- +(0, 0, 2) node [left] {$y$};

  % Upstream tracking stations
  \filldraw[station] (1, -2, -1.5) -- (1, -2, 1.5) -- (1, 2, 1.5) -- (1, 2, -1.5) -- (1, -2, -1.5);
  \filldraw[station] (2, -2, -1.5) -- (2, -2, 1.5) -- (2, 2, 1.5) -- (2, 2, -1.5) -- (2, -2, -1.5);
  \filldraw[station] (3, -2, -1.5) -- (3, -2, 1.5) -- (3, 2, 1.5) -- (3, 2, -1.5) -- (3, -2, -1.5);
  % Downstream tracking stations
  \filldraw[station] (8, -2, -1.5) -- (8, -2, 1.5) -- (8, 2, 1.5) -- (8, 2, -1.5) -- (8, -2, -1.5);
  \filldraw[station] (9, -2, -1.5) -- (9, -2, 1.5) -- (9, 2, 1.5) -- (9, 2, -1.5) -- (9, -2, -1.5);
  \filldraw[station] (10, -2, -1.5) -- (10, -2, 1.5) -- (10, 2, 1.5) -- (10, 2, -1.5) -- (10, -2, -1.5) node [below, sloped, near end] {\tiny{Tracking station}};

  % Muons
  \draw[particle] plot [smooth] coordinates {(O) (5.5, 0, 1.5) (13, 0, 0.5)} node [right] {$\mu^{-}$};
  \draw[particle] plot [smooth] coordinates {(O) (4.5, 0, -1.5) (13, 0, -1.0)} node [right] {$\mu^{+}$};

  % Hits
  % Negative muon at upstream stations
  \node[hit] at (1, 0, 0.3) {};
  \node[hit] at (1.1, 0, 0.4) {};
  \node[hit] at (3, 0, 0.9) {};
  \node[hit] at (3, 0, 1.0) {};
  \node[hit] at (4.5, 0, 1.5) {};
  \node[hit] at (4.3, 0, 1.3) {};
  \node[hit] at (4.4, 0, 1.3) {};
  % Negative muon at downstream stations
  \node[hit] at (8, 0, 1.2) {};
  \node[hit] at (8.1, 0, 1.4) {};
  % \node[hit] at (9.2, 0, 1.1) {};
  \node[hit] at (9.3, 0, 1.0) {};
  \node[hit] at (11.5, 0, 0.8) {};
  \node[hit] at (11.3, 0, 0.8) {};
  \node[hit] at (11.4, 0, 0.9) {};

  % Magnetic field vector
  \draw[thick,->] (5, 2, 1.5) -- (5, 2, 1.0) node [anchor=west] {$\vec{B}$ field} -- (5, 2, 0.5);
\end{tikzpicture}

\end{table}

\section{Branching fractions}
\label{chap:prod:syst:bf}

There is an uncertainty on the branching fractions, provided by the Particle 
Data Group~\cite{PDG2014}, which we will propagate as a systematic uncertainty 
on our measurement. The numerical values used are listed in 
\cref{tab:prod:introduction:branching_ratios}. We use these particular values 
as they are the same as those used in the \sqrtseq{13} charm production 
measurement, making the statistical treatment when computing ratios between 
centre-of-mass energies very easy (the branching fractions will cancel).

% TODO shorten this *a lot*, it's discussed in the intro and sel chapters
As this measurement does not determine the resonant \phiToKK\ contribution in 
the \DspTophipi\ sample, but instead only selects events within a 
$\pm\SI{20}{\MeV}$ window around the \Pphi mass, as described in 
Section~\ref{chap:prod:sel:offline}, a dedicated measurement by CLEO is used 
\cite{Alexander:2008aa}.
This measures the branching fraction of \DspTophipi, where a `$\phi$' is 
identified by the same $\pm\SI{20}{\MeV}$ window definition.

\section{Fit model}
\label{chap:prod:syst:fitting}

Other fit models may describe the well but give different results.
The magnitude of the difference between the cross sections computed with other 
fit models and our `nominal' model, described in \cref{chap:prod:fitting}, is 
assigned as a systematic uncertainty on the result.
This corresponds to an uncertainty on what the true generating probability 
density function is for the observed data.

The resulting relative systematic uncertainties are given in 
Tables~\ref{tab:syst:fitting:DzToKpi},
\ref{tab:syst:fitting:DpToKpipi},
\ref{tab:syst:fitting:DsTophipi}, and
\ref{tab:syst:fitting:DstToDzpi_DzToKpi} for \DzToKpi, \DpToKpipi, \DspTophipi, 
and \DstToDzpi with \DzToKpi.

\begin{table}
  \caption{%
    Relative uncertainty on the \PDzero cross-section, in \PDzero \pTy\ bins, 
    due to the arbitrary choice of fit model.
  }
  \label{tab:syst:fitting:DzToKpi}
  \centering
  % \input{tables/systematics/D0ToKpi/fit}
\end{table}

\section{Signal window}
\label{chap:prod:syst:signal_window}

The signal window that is applied to the data entering the \lnipchisq\ fit, as 
described in \cref{chap:prod:fitting}, and this has an efficiency, as 
described in \cref{chap:prod:effs:signal_window}.
The computation of this efficiency assumes the signal model is known, but the 
parameters of the model are fitted to the data, and so have an uncertainty, 
which results in a systematic uncertainty on the signal window efficiency.

This computation of the uncertainty on the signal window efficiency is given in
\cref{chap:prod:effs:signal_window}.
This uncertainty is propagated the cross-sections as the systematic 
uncertainty.

\section{Correlations and summary}
The different sources of systematic uncertainty discussed in this section can 
be grouped into different categories:
global uncertainties which are correlated between all bins and all modes, 
uncertainties which are are only correlated
between the bins of one mode and those which are uncorrelated between bins and 
modes.
\subsubsection*{Global uncertainties}
\begin{itemize}
  \item Branching ratio for \PDzero and \DstToDzpi\ for the same \PDzero final 
    state.
  \item Luminosity
  \item Tracking correction
\end{itemize}
\subsubsection*{Correlated between different bins}
\begin{itemize}
  \item Branching ratio
  \item Fit model
  \item \ac{PID} calibration sample size
  \item \ac{PID} binning
\end{itemize}
\subsubsection*{Uncorrelated}
\begin{itemize}
  \item Monte Carlo sample size
\end{itemize}
An overview of the uncertainties is given in Table~\ref{tab:sys:summary}. As we 
employ Monte Carlo
error propagation, correlations due to a common input to the measurement of 
meson species (e.g.\ they all use the same tracking correction tables and 
systematic uncertainty due to detector material modelling) are implicitly 
included in the computation (also across different centre-of-mass energies) the 
same Monte Carlo error propagation list is used.
% \chapter{Overview}
\label{chap:intro:overview}

The Large Hadron Collider is particle accelerator at CERN, a particle physics 
laboratory, in Geneva, Switzerland.
It is a ring \SI{27}{\kilo\metre} in circumference designed to accelerate two 
oppositely-circulating beams of protons up to energies of \SI{8}{\TeV}.
The beams collide into each other at four points around the ring, at each of 
which a particle detector records the resulting showers of particles.

Two of these detectors, ATLAS and CMS, are designed to study a wide range of 
phenomena, including searches for particles not seen before the \ac{LHC} such 
as the Higgs boson, while ALICE is designed to study the quark-gluon plasma, 
the primary state of matter that existed in the early universe.
This thesis is concerned with the LHCb detector, which is designed to measure 
the properties of hadrons containing charm and beauty quarks, the two heaviest 
quarks that can form bound states.

The observed imbalance of matter and anti-matter in the universe cannot be 
explained by established theoretical models, and the physics of heavy flavour, 
interactions involving charm and beauty quarks, can provide some insight in to 
possible causes.
The properties and decays of beauty hadrons have been observed to be different 
between matter and anti-matter, and similar behaviour is expected in charm 
hadrons, albeit with a smaller magnitude.
The high energy of the \ac{LHC} beams produces large samples of charm and 
beauty hadrons, allowing for precise measurements of their properties to be 
made.

This thesis will cover two different topics: physics analysis, and detector 
monitoring.
The former involves using data collected by the detector to measure specific 
physical quantities.
The first analysis presented here is a measurement of the rate of production of 
mesons containing charm quarks, performed using data taken at a proton-proton 
centre-of-mass energy of \runtwocom, presented in \cref{chap:prod}.
The second analysis is a measurement of the relative rate of the decays of 
charm and anti-charm baryons, presented in \cref{chap:cpv}.
\cref{chap:velo} describes the motivation for and implementation of a new 
software framework for monitoring the health and performance of the Vertex 
Locator sub-detector.

The design and operation of the LHCb detector plays an important part in all 
that follows, and so it shall be described in detail in \cref{chap:intro:lhcb}.
The physical context in which LHCb lives, the \ac{LHC}, will be presented 
in \cref{chap:intro:lhc}, whilst the ideological context, the \ac{SM}, will be 
presented in \cref{chap:intro:sm}.


% \section{Correlations between \comenergy and \SI{13}{\TeV} measurements}
% \label{sec:syst:13tev_correlations}

% The use of Monte Carlo error propagation allows a seamless treatment of correlations even between different years.
% Correlated systematic uncertainties are: the branching fractions, the luminosity and the tracking calibration.
% As the same numerical values for the branching fractions are used, the corresponding systematic uncertainty is fully correlated between the two years. The correlation between for the luminosity uncertainty is taken to be 32\% (luminosity working group) and the contributions to tracking due to the uncertainty on the material budget are assumed to be fully correlated.
